% \documentclass{article}
% \usepackage{graphicx} % Required for inserting images
% \usepackage{pbox}
% \usepackage{amsmath,amssymb,amsfonts,amsthm,mathptmx}
% \usepackage{stmaryrd}
% 

%%%%%% Basic Math
\newcommand{\comment}[1]{\textcolor{red}{#1}}

\newcommand{\notimplies}{%
  \mathrel{{\ooalign{\hidewidth$\not\phantom{=}$\hidewidth\cr$\implies$}}}}


\newcommand{\RN}{\mathbb{R}}
\newcommand{\QN}{\mathbb{Q}}
\newcommand{\NN}{\mathbb{N}}
\newcommand{\BN}{\mathbb{B}}
\newcommand{\ZN}{\mathbb{Z}}
\newcommand{\power}{\mathcal{P}}

\newcommand{\true}{\mathbf{true}}
\newcommand{\false}{\mathbf{false}}


\newcommand{\simplies}{\Rightarrow}
\newcommand{\simplied}{\Leftarrow}
\newcommand{\siff}{\Leftrightarrow}
\newcommand{\emptystring}{\epsilon}
\newcommand{\gfp}{\mathrm{gfp}}
\newcommand{\lfp}{\mathrm{lfp}}
\newcommand{\bottom}{\perp}

\newcommand{\valfoo}{\nu}
\newcommand{\eval}[1]{\llbracket #1 \rrbracket}
\newcommand{\interp}{\mathcal{I}}

\newcommand{\sat}{\mathbf{sat}}
\newcommand{\unsat}{\mathbf{unsat}}
%%%%%% Merge Sort

\newcommand{\mergesort}{\mathtt{mergesort}}
\newcommand{\merges}{\mathtt{merge}}
\newcommand{\odd}{\mathtt{odd}}
\newcommand{\even}{\mathtt{even}}
\newcommand{\sorted}{\mathtt{sorted}}
\newcommand{\reverseA}{\mathtt{reverse}}
\newcommand{\reverseB}{\mathtt{reverse2}}

\newcommand{\parent}{\mathtt{parent}}
\newcommand{\monkey}{\mathtt{monkey}}
\newcommand{\human}{\mathtt{human}}


\newcommand{\proofs}{\vdash}


\newcommand{\hilbert}{\mathcal{H}}
% \newcommand{\nj}{\mathcal{N}\mkern-9mu\mathcal{J}}
% \newcommand{\nk}{\mathcal{N}\mkern-6mu\mathcal{K}}
% \newcommand{\lk}{\mathcal{L}\mkern-6mu\mathcal{K}}
\newcommand{\nj}{\mathcal{NJ}}
\newcommand{\nk}{\mathcal{NK}}
\newcommand{\lk}{\mathcal{LK}}
 
\newcommand{\kripke}{m}
\newcommand{\krel}{\preceq}
\newcommand{\kinit}{w_0}
\newcommand{\kworld}{W}


\newcommand{\lif}{\mathrm{if}}
% \newcommand{\ln}{\underline{if}}


\newcommand{\tuple}[1]{\langle #1 \rangle}

\newcommand{\sif}[3]{\mathrm{if}\;#1\;\mathrm{then}\;#2\;\mathrm{else}\;#3}
\newcommand{\swhile}[2]{\mathrm{while}\;#1\;\mathrm{do}\;#2}
\newcommand{\siffi}[1]{\mathrm{if}\;#1\;\mathrm{fi}}
\newcommand{\sdo}[1]{\mathrm{do}\;#1\;\mathrm{od}}
\newcommand{\sskip}{\mathtt{skip}}
\newcommand{\sassert}{\mathtt{assert}}
\newcommand{\sfail}{\mathtt{fail}}
\newcommand{\sassume}{\mathtt{assume}}
\newcommand{\sabort}{\mathtt{abort}}
\newcommand{\ssc}{\mathtt{c}}
\newcommand{\ssb}{\mathtt{b}}
\newcommand{\sgc}{\mathtt{gc}}

\newcommand{\simp}[1]{\mathtt{#1}}
\newcommand{\sop}[2]{\langle \simp{#1},~#2 \rangle}
\newcommand{\sde}[1]{\llbracket \simp{#1} \rrbracket}
\newcommand{\sdefun}[1]{\lambda \sigma.~ #1}

\newcommand{\imp}[1]{\texttt{#1}}
\newcommand{\opsem}[2]{\langle \imp{#1},~#2 \rangle}
\newcommand{\desem}[1]{\llbracket \imp{#1} \rrbracket}
\newcommand{\defun}[1]{\lambda \sigma.~ #1}



\newcommand{\vc}{\mathrm{vc}}
\newcommand{\wlp}{\mathrm{wlp}}
\newcommand{\wsp}{\mathrm{wp}}



\newcommand{\ccs}{\mathcal{P}}

\newcommand{\coin}{\;\mathrm{coin}\,}
\newcommand{\discof}{\;\mathrm{discof}\,}
\newcommand{\reqcof}{\;\mathrm{reqcof}\,}
\newcommand{\distea}{\;\mathrm{distea}\,}
\newcommand{\reqtea}{\;\mathrm{reqtea}\,}


\newcommand{\inc}{\;\mathrm{inc}\,}
\newcommand{\dec}{\;\mathrm{dec}\,}
\newcommand{\zero}{\;\mathrm{zero}\,}
\newcommand{\Counter}{\;\mathrm{Counter}\,}
\newcommand{\semicounter}{\;\mathrm{Semicounter}\,}



\newcommand{\equiso}{\approx_{\mathrm{iso}}}
\newcommand{\equbis}{\approx_{\mathrm{bis}}}
\newcommand{\equsim}{\approx_{\mathrm{sim}}}
\newcommand{\equlin}{\approx_{\mathrm{lin}}}

\newcommand{\lequsim}{\preceq_{\mathrm{sim}}}
\newcommand{\gequsim}{\succeq_{\mathrm{sim}}}

\providecommand{\T}{T}
\DeclareMathOperator{\pre}{pre}
\DeclareMathOperator{\post}{post}
\DeclareMathOperator{\HML}{HML}
\DeclareMathOperator{\HMLlin}{HML_{\mathrm{lin}}}



\newcommand{\bev}[1]{\langle #1 \rangle}
\newcommand{\bal}[1]{[ #1 ]}

\newcommand{\lev}{\lozenge}
\newcommand{\lal}{\square}
\newcommand{\lne}{\bigcirc}
\newcommand{\lun}{\,\mathcal{U}\,}

\newcommand{\lmu}{\mu\,}
\newcommand{\lnu}{\nu\,}

\newcommand{\algo}{\mathcal{A}}


\newcommand{\tatom}{\mathrm{atom}}
\newcommand{\tcar}{\mathrm{car}}
\newcommand{\tcdr}{\mathrm{cdr}}
\newcommand{\tcons}{\mathrm{cons}}

\newcommand{\tread}{\mathrm{read}}
\newcommand{\twrite}{\mathrm{write}}


% \title{Formalisms Every Computer Scientist Should Know}
% \author{Lecture 19}
% \date{\today}

% \begin{document}

% \maketitle
% \section{}


\begin{example}
\label{cl19:ex:equ}
When are two processes equal? For example consider the following syntactic CCS expressions
\begin{align*}
    P=a(bP + c), \quad P'=bP+c, \text{and} \quad P''=c+bP
\end{align*}
To answer this we look at their semantics, which is given by LTS, e.g., 
   \begin{center}
 \begin{tikzpicture}[->, >=Stealth, node distance=2cm, thick, 
    main/.style = {draw, ellipse}, 
    dot/.style = {circle, fill, inner sep=1.5pt}]
    % Nodes
  
    \node[main] (1) at (0, 0) {\small$P$};
    \node[main] (2) at (2, 0) {\small$P'$};
    \node[main] (3) at (4, 0) {\small$0$};

    \node[dot] (4) at (7, 0) {};
    \node[dot] (5) at (9, 0) {};
    \node[dot] (6) at (11, 0) {};

    \node (0) at (5.5, 0) {$\implies$};
    % Edges
    \draw[bend left] (1) to node[midway, above] {\small$a$} (2) ;
     \draw  (2) to node[midway, above] {\small$c$} (3) ;
    \draw[bend left]  (2) to node[midway, above] {\small$b$} (1) ;

    \draw[bend left]  (4) to node[midway, above] {\small$a$} (5) ;
    \draw  (5) to node[midway, above] {\small$c$} (6) ;
    \draw[bend left]  (5) to node[midway, above] {\small$b$} (4) ;

\end{tikzpicture}
 \end{center}
 Hence, the question can be reduced to defining equivalences between LTS.
\end{example}



 
\begin{definition}[Equivalences]
    the four common equivalence relations on LTS are: 
    (i) $\equiso$ isomorphism between LTS; 
    (ii) $\equbis$ bisimilarity;
    (iii) $\equsim$ similarity;
    (iv) $\equlin$ language equivalence.
\end{definition}

\begin{theorem}
    We know that:
    \begin{align*}
        \equiso &\implies  \equbis \implies  \equsim \implies  \equlin \\
        \equlin &\notimplies  \equsim \notimplies  \equbis \notimplies  \equiso
    \end{align*}
\end{theorem}


\begin{definition}[Isomorphism]
Let $T_1 = (S_1,\mapsto)$ and $T_2 = (S_2,\mapsto)$. Let $s_1 \in S_1$ and $s_2 \in S_2$. Then $s_1 \equiso s_2$ iff there exists $h: S_1 \mapsto S_2$ such that:
\setlist{nolistsep}
\begin{enumerate}[(i), noitemsep]
    \item $h$ is a bijection (onto and 1-1);
    \item for all $s,s' \in S_1$ and all actions $a \in A$, we have $s \xrightarrow{a} s'$ iff $h(s) \xrightarrow{a} h(s')$;
    \item $h(s_1) = s_2$.
\end{enumerate}
% (i) $h$ is a bijection (onto and 1-1); (ii) for all $s,s' \in S_1$ and all actions $a \in A$, we have $s \xrightarrow{a} s'$ iff $h(s) \xrightarrow{a} h(s')$; $h(s_1) = s_2$.
\end{definition}

\begin{remark}
    Isomorphism can be seen as a structural equivalence. 
\end{remark}


\begin{example}[Ex.~\ref{cl19:ex:equ} cont.]
    It is easy to see that  $P' \equiso P''$ due to the commutativity of $+$, i.e., $P+Q=Q+P$. 
    However, this notion of equivalence is too strong to capture $P=P+P$, i.e., semantically $P$ and $P+P$ can be represented  as
    
    \begin{center}
    \begin{tikzpicture}[->, >=Stealth, node distance=2cm, thick, 
    main/.style = {draw, ellipse}, 
    dot/.style = {circle, fill, inner sep=1.5pt}]
    % Nodes

    \node[dot, label=right:{\small$s_1$}] (1) at (0, 0) {};
    \node[dot, label=right:{\small$s_1'$}] (2) at (0, -1.5) {};

    
    \node[dot, label=right:{\small$s_2$}] (4) at (5, 0) {};
    \node[dot, label=right:{\small$s_2'$}] (5) at (4, -1.5) {};
    \node[dot, label=right:{\small$s_2''$}] (6) at (6, -1.5) {};

    % Edges
  
    \draw  (1) to node[midway, right] {\small$a$} (2) ;
    \draw  (4) to node[midway, left] {\small$a$} (5) ;
    \draw  (4) to node[midway, right] {\small$a$} (6) ;
    

\end{tikzpicture}
 \end{center}
 As there is no bijection to be found the two LTS are not isomorphic. Hence, we are interested in weaker notions of similarity. 
\end{example}


\begin{definition}[Bisimilarity]
    Let $T_1 = (S,\mapsto)$. Let $s_1 \in S$ and $s_2 \in S$. Then $s_1 \equbis s_2$ iff there exists $r \subseteq S \times S$ such that:
    \setlist{nolistsep}
    \begin{enumerate}[(i),noitemsep]
        \item $ \forall s_1,s_2\in S$ if  $(s_1,s_2) \in r$, then $\forall s_1' \in S, \forall a\in A$ if $s_1 \xrightarrow{a} s_1'$ then $\exists s_2'$ such that $s_2 \xrightarrow{a} s_2'$ and $(s_1',s_2') \in r$;
        \item $\forall s_1,s_2$ if $(s_1,s_2) \in r$, then $\forall s_2'\in S, \forall a \in A$ if $s_2 \xrightarrow{a} s_2'$, then $\exists$ $s_1'$ such that $s_1 \xrightarrow{a} s_1'$ and $(s_1',s_2') \in r$; 
        \item $(s_1,s_2) \in r$.
    \end{enumerate}
\end{definition}

\begin{example}[Ex.~\ref{cl19:ex:equ} cont.]
    Continuing. While $P$ and $P+P$ are not isomorphic, they are bisimilar, i.e., we define $r\coloneqq \{(s_1, s_2), (s_1',s_2'), (s_1',s_2'')\}$
    \begin{center}
\begin{tikzpicture}[->, >=Stealth, node distance=2cm, thick, 
    main/.style = {draw, ellipse}, 
    dot/.style = {circle, fill, inner sep=1.5pt}]
    % Nodes

    \node[dot, label=right:{\small$s_1$}] (1) at (0, 0) {};
    \node[dot, label=right:{\small$s_1'$}] (2) at (0, -1.5) {};

    \node[dot, label=right:{\small$s_2$}] (4) at (5, 0) {};
    \node[dot, label=right:{\small$s_2'$}] (5) at (4, -1.5) {};
    \node[dot, label=right:{\small$s_2''$}] (6) at (6, -1.5) {};

    % Edges
    \draw  (1) to node[midway, right] {\small$a$} (2) ;
    \draw  (4) to node[midway, left] {\small$a$} (5) ;
    \draw  (4) to node[midway, right] {\small$a$} (6) ;

    % Boxes around top and bottom nodes
    \node[draw, dashed, fit=(1)(4), label=above:{\small$(s_1, s_2)$}, inner xsep=5mm, inner ysep=2mm] {};
    \node[draw, dashed, fit=(2)(5)(6), label=below:{\small$(s_1', s_2'), (s_1', s_2'')$}, inner xsep=5mm, inner ysep=2mm] {};

\end{tikzpicture}
 \end{center}
\end{example}

\begin{remark}
    Both the identity and all graph isomorphism are bisimulations. The converse does not hold. 
    Computing isomorphisms is hard, i.e., it is somewhere between $\mathrm{P}$ and $\mathrm{NP}$. Whereas computing bisimulations is relatively easy, i.e., $\mathcal{O}(m\log n)$ where $n$ is the number of states and $m$ the number of edges. 
\end{remark}

% \begin{itemize}
%     \item 
%     \item the opposite of the above is not true
%     \item computing isomorphisms is hard 
%     \item computing bisimilarity is "easy" $O(m \log n)$ where $m$ is the number of edges and $n$ is number of states.
% \end{itemize} 

\begin{theorem}
    \label{cl19:thrm:bis}
    Let $r \subseteq S\times S$ and let 
    \begin{align*}
        C_{(i)}(r,s_1, s_2)&\coloneqq \forall s_1' \forall a.\;  s_1 \xrightarrow{a} s_1' \simplies \exists s_2'.\;  s_2 \xrightarrow{a} s_2' \wedge r(s_1',s_2') \\
        C_{(ii)}(r,s_1, s_2)&\coloneqq \forall s_2' \forall a .\; s_2 \xrightarrow{a} s_2' \simplies \exists s_1'.\;  s_1 \xrightarrow{a} s_1' \wedge r(s_1',s_2') \}
    \end{align*}
    We define $F(r) = \{(s_1,s_2) | C_{(i)}(r,s_1, s_2) \land C_{(ii)}(r,s_1, s_2)\}$. 
    Then $\equbis = \gfp F$.
\end{theorem}

\begin{remark}
    From Theorem \ref{cl19:thrm:bis} we can derive a simple algorithm for computing $\equbis$ by starting from $S\times S$ and iteratively refining the relation. 
    \begin{align*}
        && \equbis^{0} &= S\times S \\
       &\supseteq & \equbis^{1} &= F(\equbis^{0}) =  \{(s_1,s_2) | C_{(i)}(\equbis^{0},s_1, s_2) \land C_{(ii)}(\equbis^{0},s_1, s_2)\}\\ 
       & \supseteq & \equbis^{2} &= F(\equbis^{1}) =  \{(s_1,s_2) | C_{(i)}(\equbis^{1},s_1, s_2) \land C_{(ii)}(\equbis^{1},s_1, s_2)\}\\ 
        &\supseteq & \vdots \\
        & \supseteq & \equbis &= \gfp F(S\times S)
    \end{align*}
    This is called the simple fixed-point approximation algorithm (aka. partition refinement): $O(m\cdot n)$
\end{remark}


\begin{example}
    Consider the following LTS
    \begin{center}
 \begin{tikzpicture}[->, >=Stealth, node distance=2cm, thick, main/.style = {draw, circle},
    dot/.style = {circle, fill, inner sep=1.5pt}]
    % Nodes
    \node[main] (s1) at (0, 0) {\small$s_1$};
    \node[main] (s2) at (3, 0) {\small$s_2$};
    \node[main] (t1) at (0, -1.5) {\small$t_1$};
    \node[main] (t2) at (2, -1.5) {\small$t_2$};
    \node[main] (t3) at (4, -1.5) {\small$t_3$};
    \node[main] (v1) at (-1, -3) {\small$v_1$};
    \node[main] (v2) at (1, -3) {\small$v_2$};
    \node[main] (v3) at (4, -3) {\small$v_3$};

    \draw (s1) to node[midway, right] {\small$ a $} (t1) ;
    \draw (t1) to node[midway, right] {\small$ b $} (v1) ;
    \draw (t1) to node[midway, right] {\small$ c $} (v2) ;
    \draw[bend left] (v1) to node[midway, right] {\small$ d $} (s1) ;

    \draw (s2) to node[midway, right] {\small$ a $} (t2) ;
    \draw (s2) to node[midway, right] {\small$ b $} (t3) ;
    \draw (t2) to node[midway, right] {\small$ c $} (v2) ;
    \draw (t3) to node[midway, right] {\small$ b $} (v3) ;
    \draw[bend right=90] (v3) to node[midway, right] {\small$ d $} (s2) ;
   
    % \node[main] (2) at (2, 0) {\small$V'$};
    % \node[main] (3) at (5, 1) {\small$\distea!V$};
    % \node[main] (4) at (5, -1) {\small$\discof!V$};

    % % Edges
    % \draw  (1) to node[midway, above] {\small$\coin?$} (2) ;
    % \draw  (2) to node[midway, above] {\small$\reqtea?$} (3) ;
    % \draw  (2) to node[midway, below] {\small$\reqcof?$} (4);
    % \draw[bend right] (3) to node[midway, below] {\small$\distea!$} (1) ;
    % \draw[bend left] (4) to node[midway, above] {\small$\discof!$} (1);
\end{tikzpicture}
 \end{center}
 Then applying the fixed-point approximation algorithm we get 
\begin{align*}
       \equbis^{0} &= S\times S \\
       \equbis^{1} &=   \mathrm{id} \cup \{ (v_1,v_2), (v_1, v_3), (v_2, v_1), (v_2, v_3), (v_3,v_1),(v_3, v_2)\}\\ 
      & \vdots\\
      \equbis &=   \mathrm{id} \cup \{(v_1, v_2), (v_2, v_1)\}
\end{align*}
where $\mathrm{id}\coloneqq \{(s_1,s_1), (s_2,s_2), (t_1,t_1), (t_2,t_2), (t_3,t_3), (v_1,v_1), (v_2,v_2), (v_3,v_3)
\}$ is the identity relation.
\end{example}
% \textbf{Algorithm.} Let $r \in S\times S$ and define $F(r) = \{(s_1,s_2) | \textit{1. and 2. from definition of bisimilarity}\}$. 

% It holds that $\sim_{bis} = \mathit{gfp}~ F$. 

% \begin{align*}
%     \sim_{bis}^0 &= S \times S \\
%     \sim_{bis}^1 &= F(\sim_{bis}^0) = \{(s_1,s_2)| & \forall s_1', s_1 \xrightarrow{a} s_1' \Rightarrow \exists s_2'; s_2 \xrightarrow{a} s_2' \wedge \sim_{bis}^0(s_1',s_2') \\ 
%     & & \wedge  \forall s_2', s_2 \xrightarrow{a} s_2' \Rightarrow \exists s_1'; s_1 \xrightarrow{a} s_1' \wedge \sim_{bis}^0(s_1',s_2') \}
% \end{align*}
% This is called the simple fixed-point approximation algorithm (aka. partition refinement): $O(m.n)$

\begin{definition}[Simulation]
    \label{cl19:def:sim}
    For $s_1,s_2 \in S$, $s_2$ simulates $s_1$, i.e., $s_1 \lequsim s_2$, iff $\exists r \in S \times S$ such that:
    \setlist{nolistsep}
    \begin{enumerate}[(i),noitemsep]
        \item $\forall s_1,s_2$ if $s_1 \lequsim  s_2$ then, $\forall s_1' \forall a$  if $s_1 \xrightarrow{a} s_1'$ then $\exists s_2'$ such that $s_2 \xrightarrow{a} s_2'$ and $s_1' \lequsim  s_2'$;
        \item $(s_1,s_2) \in r$
    \end{enumerate}
\end{definition}

\begin{remark}
    Note that in Definition \ref{cl19:def:sim} the second condition in bisimilarity is not present in the definition of simulation. (Example on difference on blackboard)
    Hence, it is not symmetric.
\end{remark}

\begin{example}
    The asymmetry of $\lequsim$ can be observed in the following example. The right structure simulates the left, but the inverse does not hold.
    \begin{center}
    \begin{tikzpicture}[->, >=Stealth, node distance=2cm, thick, 
    main/.style = {draw, ellipse}, 
    dot/.style = {circle, fill, inner sep=1.5pt}]
    % Nodes

    \node[dot, label=right:{\small$s_1$}] (1) at (0, 0) {};
    \node[dot, label=right:{\small$s_1'$}] (2) at (0, -1.5) {};


    \node (b) at (2,-0.25) {\large$\lequsim$};
    \node (a) at (2, -1.25) {\large$\not\gequsim$};
    
    \node[dot, label=right:{\small$s_2$}] (4) at (5, 0) {};
    \node[dot, label=right:{\small$s_2'$}] (5) at (4, -1.5) {};
    \node[dot, label=right:{\small$s_2''$}] (6) at (6, -1.5) {};

    % Edges
  
    \draw  (1) to node[midway, right] {\small$a$} (2) ;
    \draw  (4) to node[midway, left] {\small$a$} (5) ;
    \draw  (4) to node[midway, right] {\small$a$} (6) ;
    

\end{tikzpicture}
 \end{center}
\end{example}


\begin{definition}[Similarity]
    Two states $s_1,s_2$ are called similar or $s_1 \equsim s_2$ iff (i) $s_1 \lequsim  s_2$ and (ii) $s_2 \lequsim  s_1$.
\end{definition}

\begin{remark}
    Bisimilarity implies similarity, but the opposite direction is not generally true.
    Similarity is a greatest fixed-point and can be computed via approximation sequence $\equsim^0, \equsim^1, \dots$ in $\mathcal{O}(m\cdot n)$.
\end{remark}

\begin{example}
    The following demonstrates that similartiy is a weaker form of equivalence than bisimulation. 
        \begin{center}
\begin{tikzpicture}[->, >=Stealth, node distance=2cm, thick, 
    main/.style = {draw, circle}, dot/.style = {circle, fill, inner sep=1.5pt}]
    
    % Nodes with labels to the right
    \node[dot, label=right:{\small$s_1$}] (s1) at (0, 0) {};
    \node[dot, label=left:{\small$s_2$}] (s2) at (5, 0) {};
    \node[dot, label=right:{\small$t_1$}] (t1) at (0, -1.5) {};
    \node[dot, label=right:{\small$t_1$}] (t2) at (4, -1.5) {};
    \node[dot, label=right:{\small$t_2$}] (t3) at (6, -1.5) {};
    \node[dot, label=right:{\small$u$}] (v1) at (-1, -3) {};
    \node[dot, label=right:{\small$u$}] (v2) at (1, -3) {};
    \node[dot, label=left:{\small$u$}] (v3) at (3, -3) {};
    \node[dot, label=left:{\small$u$}] (v4) at (5, -3) {};
    \node[dot, label=left:{\small$u$}] (v5) at (6, -3) {};

    \node (b) at (2,-0.5) {\large$\not\equbis$};
    \node (b) at (2,-1.5) {\large$\lequsim$};
    \node (b) at (2,-2.5) {\large$\equsim$};

    % Edges
    \draw (s1) to node[midway, right] {\small$ a $} (t1) ;
    \draw (t1) to node[midway, right] {\small$ b $} (v1) ;
    \draw (t1) to node[midway, right] {\small$ c $} (v2) ;


    \draw (s2) to node[midway, right] {\small$ a $} (t2) ;
    \draw (s2) to node[midway, right] {\small$ b $} (t3) ;
    \draw (t2) to node[midway, right] {\small$ b $} (v3) ;
    \draw (t2) to node[midway, right] {\small$ c $} (v4) ;
    \draw (t3) to node[midway, right] {\small$ b $} (v5) ;


\end{tikzpicture}
 \end{center}
 in this transition system we have 
 \begin{align*}
     \equbis &= \{(s,s) \mid s\in S\} = \mathrm{id}\\
     \lequsim &= \mathrm{id} \cup \{ (t_2,t_1), (s_1,s_2), (s_2, s_1)\}   \\
      \equsim &=  \mathrm{id} \cup \{ (s_1,s_2), (s_2, s_1)\} \\
 \end{align*}
\end{example}


\begin{theorem}
    \label{cl19:thrm:CCS}
    Bisimilarity is a congruence for CCS, i.e., (i) bisimilarity is an equivalence and (ii) for all CCS processes $P,Q,R$, if $P \equbis Q$ then $R[P] \equbis R[Q]$, e.g., $R+P \equbis R+Q$ or  $R \parallel P \equbis R \parallel Q$
\end{theorem}

\begin{exercise}
    Proof (ii) in Theorem~\ref{cl19:thrm:CCS}.
\end{exercise}
% \end{document} 
