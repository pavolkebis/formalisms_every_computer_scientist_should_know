
\subsection{Formal Systems}

\begin{definition}[Formal System]
    A \emph{formal system} $F$ is a set of rules.
    Rule is a finite set of (formulas) premises $p_0, \dots, p_k$ and (a formula called) conclusion $c$.
    An axiom is a rule without premises.
\end{definition}


\begin{remark}
We usually have infinitely many rules but only finitely many different rule schemata.
For example, schema 
\begin{prooftree}
   \AxiomC{}
   \UnaryInfC{$\varphi \simplies \varphi$}
\end{prooftree}
gives infinitely many rules, e.g., 
\begin{prooftree}
   \AxiomC{}
   \UnaryInfC{$p_3 \simplies p_3$}
\end{prooftree}
\end{remark}

\begin{definition}[Proof: Linear View]
A \emph{proof} (derivation) is a finite sequence of formulas $\varphi_0, \dots, \varphi_n$ such that every formula in the sequence is (i) either an axiom (which can be viewed as a special case of the following), or (ii) the conclusion of a rule whose premises occur earlier in the sequence.
\end{definition}

\begin{remark}
    Linear view is usually easier for proving meta theorems.
    Tree view (inductive definition) is usually better in practice.
\end{remark}

\begin{definition}[Theorem]
    A theorem can be defined (i) syntactically or (ii) semantically.
    The formula $\varphi$ is a \emph{theorem}, 
    (i) (in the formal system $F$) if it has a proof in a formal system $F$, denoted as $\proofs_F \varphi$;
    (ii) if $\varphi$ is valid, denoted as $\models  \varphi$.
\end{definition}


\begin{remark}[Logic]
    Formal system equipped with semantics is called a logic.
    The core problem in logic is establishing $\proofs \varphi$ iff $\models \varphi$. We speak of (i) \emph{soundness} if $\proofs \varphi \implies \models  \varphi$ and of (ii) \emph{completeness} if $ \models  \varphi\implies\proofs \varphi$.
\end{remark}


\begin{definition}[Soundness and Completeness]
Let $F$ be a formal system, let $R$ be a rule of this formal system, let $L$ be a logic, and let $\psi$ be a formula then:
    \begin{align*}
        \text{$R$ is \emph{sound}} &\iff \text{if all premises of $R$ are valid, then its conclusion is valid;} \\
        \text{$F$ is \emph{sound}} &\iff \text{all rules are sound (or every theorem is valid);} \\
        \text{$F$ is \emph{complete}} &\iff  \text{every valid formula is a theorem;} \\
        \text{$F$ is \emph{consistent}} &\iff  \not\proofs_F \bot \text{ (or there exists a formula that is not a theorem);}\\
        \text{$R$ is \emph{derivable} in $F$} &\iff \text{for all formulas $\varphi$, $\proofs_{F \cup \{R\}} \varphi$ iff $\proofs_F \varphi$}; \\  \text{$\psi$ is \emph{expressible} in $L$} &\iff \text{there exists a formula $\chi$ of $L$ s.t., for all $\valfoo$, $[[ \psi ]]_{\valfoo} = [[ \chi ]]_{\valfoo}$}; \\ 
    \end{align*}
\end{definition}

\begin{example}
    For example $\varphi_1 \wedge \varphi_2$ is expressible using only $\neg$ and $\vee$ (de Morgan) as $\psi = \neg(\neg\varphi_1 \wedge \neg\varphi_2)$.
\end{example}
% Formal system $F$ is sound iff all rules are sound (or equivalently, every theorem is valid).
% Formal system $F$ is complete iff every valid formula is a theorem.
% Formal system $F$ is consistent unless $\proofs \bot$ (or equivalently, there exists a formula that is not a theorem).
% Rule $R$ is derivable in $F$ iff [for all formulas $\varphi$, $\underset{F \cup \{R\}}\proofs \varphi$ iff $\underset{F}\proofs \varphi$].
% Rule $R$ is admissible in $F$ iff $F \cup \{R\}$ is still consistent.
% Formula $\varphi$ is expressible in a logic $L$ iff [there exists a formula $\psi$ of $L$ such that, for all interpretations $v$, $[[ \varphi ]]_v = [[ \psi ]]_v$.

\begin{remark}
    We can enumerate all theorems by systematically enumerating all possible proofs. The proof is a witness for validity.
    \begin{itemize}
        \item Sound formal system $\implies$ sound procedure for validity (but not necessarily complete). 
        \item Sound and complete formal system $\implies$ sound semi-complete procedure for validity (may not terminate on inputs that represent a formula that is not valid).
    \end{itemize}
    To get a decision procedure (sound and complete procedure for validity), we need both: (i) sound complete formal system for validity, and
    (ii) sound complete formal system for satisfiability (to define a formal system for satisfiability.
    Those are formal systems systems where valid formulas are replaced by satisfiable judgments. Then all axioms are satisfiable and all rules go from satisfiables to satisfiable.
    If we have both (i) and (ii) then for every input $\varphi$, one of them will eventually terminate and we can conclude either $\varphi$ is valid, or $\neg\varphi$ is satisfiable (which means that $\varphi$ is not valid). 
    This is only possible the set is decidable, i.e., both the set and its complement are recursively-enumerable implying that the set is recursive (decidable).
\end{remark}


\subsection{Hilbert formal system $\hilbert$ for propositional logic}

\begin{definition}
     The Hilbert system $\hilbert$ uses connectives $\simplies$ and $\neg$ only, has three axioms
     \begin{align*}
         (K) \quad &\quad \varphi\simplies\psi\simplies\varphi \\
         (S) \quad &\quad (\varphi\simplies\psi\simplies\chi)\simplies((\varphi\simplies\psi)\simplies(\varphi\simplies\chi)) \\
         (ex \; middle) \quad &\quad (\neg\varphi\simplies\neg\psi)\simplies(\psi\simplies\varphi)
     \end{align*}
     and one rule, called \emph{modus ponens} (MP)
    \begin{prooftree}
        \AxiomC{$\varphi$}
        \AxiomC{$\varphi \simplies \psi$}
        \RightLabel{\small (MP)}
        \BinaryInfC{$\psi$}
    \end{prooftree}
\end{definition}


\begin{example}
    The proof of $\varphi \simplies \varphi$ in Hilbert system:
    \begin{align*}
(K) \quad & \varphi \simplies (\psi \simplies \varphi) \simplies \varphi\\
(S)  \quad &(\varphi \simplies (\psi \simplies \varphi) \simplies \varphi) \simplies ((\varphi \simplies \psi \simplies \varphi) \simplies (\varphi \simplies \varphi))\\
(MP) \quad & (\varphi \simplies \psi \simplies \varphi) \simplies (\varphi \simplies \varphi)\\
(K)  \quad &\varphi \simplies \psi \simplies \varphi\\
(MP) \quad &  \varphi \simplies \varphi\\
    \end{align*}
\end{example}

\begin{theorem}[Soundness and Completeness]
  The formal system $\hilbert$ is sound and complete for propositional logic.
\end{theorem}

\begin{corollary}
    The formal system $\hilbert$ is consistent for propositional logic.
\end{corollary}

\begin{theorem}[Deduction Theorem]
    $\Gamma \proofs \varphi \simplies \psi$ iff $\Gamma, \varphi \proofs \psi$ where $\Gamma \proofs \varphi$ denotes $\underset{F \,\cup\, \Gamma}\proofs \varphi$, i.e., the set of formulas $\Gamma$ is used as added axioms.
\end{theorem}
\begin{proof}
    ``$\implies$'': One application of modus ponens.\\
    ``$\impliedby$'': 
Assume $\psi$ has a proof $\pi$ using axioms $\Gamma$, $\varphi$, (K), (S), (ex). Show that $\varphi \simplies \psi$ has a proof $\pi'$ using $\Gamma$, (K), (S), (ex). Proof by induction on length $n$ of $\pi$.
\begin{itemize}
    \item Case $n=1$: $\psi$ must be an axiom.
Either $\psi \in \Gamma \cup \{\textup{K}, \textup{S}, \textup{em}\}$ so we prove it by (K),
or $\psi = \varphi$ so we use $\proofs \varphi \simplies \varphi$ as derived above.
\item Case $n>1$: $\psi$ is the result of an application of modus ponens.
We have $\chi$ and $\chi \simplies \psi$, both of which were derived from $\Gamma, \varphi$ in fewer steps.
Induction hypothesis gives us $\Gamma \proofs \varphi \simplies \chi$ and $\Gamma \proofs \varphi \simplies \chi \simplies \psi$.
We use (S) in the form $(\varphi \simplies \chi \simplies \psi) \simplies (\varphi \simplies \chi) \simplies (\varphi \simplies \psi)$ and apply modus ponens twice,
resulting in $\varphi \simplies \psi$ derived from $\Gamma$ only.
\end{itemize}
\end{proof}

