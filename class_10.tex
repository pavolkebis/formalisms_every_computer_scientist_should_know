\subsection{First Order Resolution: Classical Automated Theorem Proving} 



\begin{definition}[First Order Resolution]
    To proof $\models \varphi $ for a closed $\varphi $:
\begin{enumerate}
    \item Negate. Show $\neg \varphi $ is unsat.
    \item Bring $\neg \varphi $ into "prenex" form (see Remark \ref{cl10:rem:prenex}).
    \item Bring $\gamma$ into CNF.
    \item "Skolemization" gets rid of $\exists$ (see Remark \ref{cl10:rem:skolem}). 
    \item Drop $\forall$.
    \item See Theorem \ref{cl10:thrm:unifier}
\end{enumerate}
\end{definition}


\begin{remark}[Prenex]
\label{cl10:rem:prenex}
A formula is in prenex normal all quantifiers are in the front, e.g., $\exists x. \varphi (x) \land \gamma \siff \exists x (\varphi (x) \land \gamma)$.
\end{remark}


\begin{remark}[Skolemization]
\label{cl10:rem:skolem}
    Consider that  while for $\exists x.\varphi (x)$ is sat iff $\varphi (\hat{x})$ is sat, we have that $\forall y \exists x \varphi (x)$  is sat iff $\varphi (f(y))$ is sat where $f(y)$ is a new function symbol. Thus,for each $\exists x$ within the scope of a universally quantified variables $y_1, \cdots,y_n$, drop $\exists x$ and replace each bound occurrence of x by $f(y_1,\cdots,y_n)$ where f is a new n-ary function symbol. E.g. 
        \begin{align*}
            \forall x, \exists y \forall z_1,z_2 \exists y . \varphi  (x,y,z_1,z_2, u) \rightarrow \forall x \forall z_1, z_2. \varphi  (x, f(x), z_1, z_2, g(x, z_1,z_2)).
        \end{align*}
\end{remark}

\begin{theorem}[Unifier]
    \label{cl10:thrm:unifier}
    For an interpretation $\interp$, clauses $C_1$ and $C_2$, and literals $\ell $ and $\ell '$. If $\interp \models C_1[\ell]$ and $\interp \models C_2[\neg \ell']$ and $\ell , \ell'$ are "unifiable" (i.e, there is a substitution, which is a function from variables to terms, that when applied to $\ell $ and $\ell '$, makes them equal), then $\interp \models X_1\theta [\bot] \lor C_2\theta [\bot]$ where $\theta$ is the most general unifier of $\ell $ and $\ell '$.
\end{theorem}


\begin{example}
        For example for a variable $x$ and a constant $S$, we have that $m(x)$ and $\neg m(S)$ are unifiable by replacing $x $ by $s$. The purpose is to make literals the same.
\end{example}

\subsection{First order theories} 


\begin{remark}
    Theories defined by (i) signature (set of functions + predicate symbols) and (ii) either by a r.e. set of closed formulas called axioms $A$ (set of all $A$-valid closed formulas) or by a specific "intended" interpretation $\interp$ (true formulas in $\interp$).
\end{remark}

\begin{definition}
    A theory $T$ is a r.e. set of closed formulas (there are no free variables) that are closed under $\models$, i.e 
     $T \models \varphi $  iff $\forall \interp ((\forall \psi \in T.  \interp \models \psi) \simplies  \interp \models \varphi )$. 
\end{definition}
Formally, 

\begin{definition}
    We say that 
\begin{itemize}
    \item $\varphi $ is $T$-valid iff $T \models \varphi $
    \item $\varphi $ is $T$-satisfiable iff $\exists  \interp.  \interp$ T-model $\varphi $.
    \item $ \interp$ is a $T$-model iff $\forall \psi \in T.  \interp \models \psi$.
    \item $\varphi , \psi$ are $T$-equivalent iff $\varphi ,\psi$ have truth value in all $T$-models.
\end{itemize}
\end{definition}

\begin{remark}
    \label{cl10:rem:theory}
    Therefore, $T$ is either (i) the set of all $A$-valid closed formulas or (ii) $T$ is the set of all formulas true in $\interp$.
\end{remark}


\begin{definition}
    The theory $T$ is: 
    \begin{itemize}
        \item   \emph{consistent} iff $\exists \interp. \forall \psi \in T$, $\interp\models \psi$ (always the case for (ii) in Remark 
        \ref{cl10:rem:theory}); 
        \item \emph{complete} iff for all closed formulas $\psi$ of the signature, either $T \models \psi$ or $T \models \neg \psi$ (always the case for (ii) in Remark \ref{cl10:rem:theory});
        \item  $T$ is \emph{decidable} iff the problem "given $\psi$, is $T \models \psi$" is decidable.
    \end{itemize}
\end{definition}


\begin{remark}
    Common theories are the theory of:
    (i)  propositional logic with quantifiers (QBF);
    (ii) equality with uninterpreted functions and predicates;
    (iii) the integers, either Peano- or Pressburger arithmetic;
    (iv) the reals over real closed fields; 
    (v) the rational numbers with linear arithmatic;
    (vi) lists and arrays.
\end{remark}