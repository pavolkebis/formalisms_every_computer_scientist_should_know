

\begin{definition}[Interpretation]
    Interpretation $\interp$ in first-order structure: 
    \begin{itemize}
        \item[(i)] Domain $D_{\interp}$
        \item[(ii)] for each $n$-ary function symbol $f \in F$, $\eval{f}_{\interp}: D_{\interp}^n \to D_{\interp}$
         \item[(iii)]  for each $n$-ary predicate symbol $p \in P$, $\eval{p}_{\interp}: D_{\interp}^n \to B = \{true,false\}$
        \item[(iv)]  "context" (environment) for each (free) variable $x \in X$, $\eval{x}_{\interp} \in D_{\interp}$
    \end{itemize}
    Formula $\varphi$ is closed if it has no free variables.
\end{definition}


\begin{definition}[Semantics]
Given an interpretation $\interp$, a term $t$ and a formula $\varphi$ have the following meaning: 
\begin{align*}
     \eval{t}_{\interp} &\coloneqq \begin{cases}
         \eval{x}_{\interp} & t=x  \\
         \eval{f}_{\interp}([ \eval{t_1}_{\interp},  \eval{t_2}_{\interp}, \dots,  \eval{t_n}_{\interp}) & t=f(t_1, \dots, t_n)
    \end{cases} \\
     [\varphi]_{\interp} &\coloneqq \begin{cases}
        \eval{x}_{\interp} & t=x  \\
        \eval{p}_{\interp}(\eval{t_1}_{\interp}, \eval{t_2}_{\interp}, \dots, \eval{t_n}_{\interp}) & \varphi = p(t_1, \dots, t_n)
    \end{cases}\\
     \eval{\varphi_1 \simplies \varphi_2}_{\interp} &\coloneqq \true \; \textit{ iff } \; \eval{\varphi_1}_{\interp} = \false \textit{ or } \eval{\varphi_2}_{\interp} = \true \\
     \eval{\forall x. \varphi}_{\interp} &\coloneqq \true \; \textit{ iff for all } \;  d \in D_{\interp}, \eval{\varphi}_{I[x \to d]}=\true \\ 
     \eval{\exists x. \varphi}_{\interp} &\coloneqq \true \; \textit{ iff for some } \; d \in D_{\interp}, \eval{\varphi}_{I[x \to d]}=\true
\end{align*}
\end{definition}

\begin{definition}[Satisfiable and Valid]
    Let $\varphi$ be a first-order formula. The formula $\varphi$ is: 
     (i)\emph{satisfiable} iff it is true for \emph{some} interpretation $\interp$, i.e., $\exists \interp . \models \varphi$.
    (ii) \emph{valid} iff it is true for \emph{all} interpretations $\interp$, i.e., $\forall \interp . \models \varphi$.
\end{definition}


\begin{definition}[Post Correspondence Problem]
    PCP (Post Correspondence Problem): Given a finite set $S$ of dominoes $\frac{s}{t}$ where $s,t \in \{0,1\}^*$. Is there a finite sequence $\frac{s_1}{t_1}, \dots \frac{s_n}{t_n}$ of (possibly repeating) dominoes from $S$ such that the 
$$s_1 \cdot s_2 \cdot \dots \cdot s_n = t_1 \cdot t_2 \dots \cdot t_n$$
\end{definition}

\begin{theorem}
    The PCP problem is undecidable.
\end{theorem}


\begin{theorem}[Compactness] 
The set of formulas $\Gamma$ is satisfiable iff every finite subset of $\Gamma$ is satisfiable.
\end{theorem}


\begin{theorem}[Lowenheim-Skolem] 
If a set $\Gamma$ of formulas is satisfiable then $\Gamma$ is satisfiable by an interpretation with countable domain. 
\end{theorem}


\begin{theorem}[Undecidability]
    Both the validity and satisfiability problems for FOL are undecidable.
\end{theorem}
\begin{proof}
   Let $F = \{e,f_0, f_1\}$ where $e$ is 0-ary and $f_0,f_1$ are unary. With those functions we can encode the string $011$ as $f_1(f_1(f_0(e)))$.  
    Let $P = \{p\}$ where $p$ is a binary predicate. Intuitively, the domino $\frac{s}{t}$ is represented by $p(s,t)$. 
    Given an instance $R\coloneqq \{r_1, \dots, r_k\}$ of PCP we define the formulas
    \begin{align*}
        \varphi_1 &\coloneqq \bigwedge_{1 \leq i \leq k} P(s_{\interp}(e), t_{\interp}(e)) \\
        \varphi_2 &\coloneqq \forall v,w. (p(v,w) \simplies \bigwedge_{1 \leq i \leq k} p(s_{\interp}(v), t_{\interp}(w))) \\ 
        \varphi_3 &\coloneqq \exists z. p(z,z)
    \end{align*}
    It remains to be shown that the formula $\varphi_R = (\varphi_1 \land \varphi_2) \simplies \varphi_3$ is valid iff the answer to $R$ is yes.
\end{proof}

\begin{exercise}
	 Use PCP to show satisfiability of FOL is undecidable. 
\end{exercise}

\subsection{Three Sound and Complete Proof Systems}


\begin{definition}[Hilbert]
We extend the Hilbert system $\hilbert$ by adding the following axioms
\begin{align*}
    (1) \quad & \quad (\forall x. \varphi) \simplies \varphi[x\coloneqq t] \\
    (2) \quad & \quad \forall x. (\varphi \simplies \psi) \simplies (\forall x. \varphi) \simplies (\forall x. \psi) \\ 
    (3) \quad & \quad \varphi \simplies \forall x. \varphi \quad \text{provided $x$ is not free in $\varphi$}
\end{align*}
We denote the safe replacement of every free variable $x$ in $\varphi$ by $t$ as $\varphi[x\coloneqq t]$
\end{definition}



\begin{definition}[Genzen]
We extend the Genzen system $\lk$ by adding the following rules for the $\forall$-quantifier
\begin{center}
    \AxiomC{$\Gamma, \varphi[x\coloneqq t] \proofs \Delta$}
    \RightLabel{\small ($\forall$-elim)}
    \UnaryInfC{$\Gamma, \forall x. \varphi \proofs \Gamma$}
    \DisplayProof 
    $\quad$
    \AxiomC{$\Gamma \proofs \Delta, \varphi[x\coloneqq y]$}
    \RightLabel{\small ($\forall$-intro)}
    \UnaryInfC{$\Gamma \proofs \Delta, \forall x. \varphi$}
    \DisplayProof 
\end{center}
and for the $\exists$-quantifier
\begin{center}
    \AxiomC{$\Gamma, \varphi[x\coloneqq y]$}
    \RightLabel{\small ($\exists$-elim)}
    \UnaryInfC{$\Gamma, \exists x. \varphi \proofs \Delta$}
    \DisplayProof 
    $\quad$
        \AxiomC{$\Gamma \proofs \Delta, \varphi[x\coloneqq t]$}
    \RightLabel{\small ($\exists$-intro)}
    \UnaryInfC{$\Gamma \proofs \Delta, \exists x. \varphi$}
    \DisplayProof 
\end{center}
where $y$ is a new (fresh) variable.
\end{definition}


\begin{definition}[Natural Deduction]
    We extend the Natural Deduction system $\nk$ (and $\nj$) by adding the following rules for the $\forall$-quantifier
\begin{center}
    \AxiomC{$\forall x. \varphi$}
    \RightLabel{\small ($\forall$-elim)}
    \UnaryInfC{$\varphi[x\coloneqq t]$}
    \DisplayProof 
    $\quad$
    \AxiomC{$\varphi[x\coloneqq y]$}
    \RightLabel{\small ($\forall$-intro)}
    \UnaryInfC{$\forall x. \varphi$}
    \DisplayProof 
\end{center}
and for the $\exists$-quantifier
\begin{center}
    \AxiomC{$\exists x . \varphi$}
            \AxiomC{$\varphi[x\coloneqq y]$}
            \noLine
        \UnaryInfC{$\vdots$}
        \noLine
    \UnaryInfC{$\psi$}
    \RightLabel{\small ($\exists$-elim)}
    \BinaryInfC{$\psi$}
    \DisplayProof 
    $\quad$
        \AxiomC{$\varphi[x\coloneqq t]$}
    \RightLabel{\small ($\exists$-intro)}
    \UnaryInfC{$\exists x. \varphi$}
    \DisplayProof 
\end{center}
where $y$ is a new (fresh) variable.
\end{definition}



\begin{exercise} 
Prove de Morgan, i.e., $ \forall x. \varphi \Leftrightarrow \neg \exists x. \neg \varphi$, for quantifiers in all three systems.
\end{exercise}
