
% \section{Tools}
% \begin{itemize}
%   \item Lean or Coq 
%   \item CVC5 or Z3 
%   \item possibly a model checker 
%   \item professional version of ChatGPT
% \end{itemize}

% \section{Syllabus}
% \begin{enumerate}
%   \item MATH (``Informalism'')
%   \begin{itemize}
%     \item proofs (natural deduction)
%     \item fixpoints (induction, coinduction)
%   \end{itemize}
%   \item DECLARATIVE LOGIC
%   \begin{itemize}
%     \item syntax (rules) vs. semantics (models)
%     \item propositional, predicate, modal logic 
%     \item decision procedures (SAT, SMT)
%   \end{itemize}
%   \item FUNCTIONS
%   \begin{itemize}
%     \item $\lambda$ calculus, typed $\lambda$
%     \item SOS (structured operational semantics), rewriting
%     \item ``propositions-as-types'' (connection to logic)
%   \end{itemize}
%   \item PROCESSES (CONCURRENT)* 
%   \begin{itemize}
%     \item CCS, Petri nets
%     \item (bi-) simulation
%   \end{itemize}
%   \item CIRCUITS*
%   \begin{itemize}
%     \item boolean, sequential, dataflow (Kahn nets)
%     \item interfaces
%   \end{itemize}
%   \item STATE TRANSITION SYSTEMS*
%   \begin{itemize}
%     \item ($\omega$-) automata, games, timed, probabilistic, pushdown
%     \item programs, Turing machines 
%     \item grammars (Chomsky hierarchy)
%   \end{itemize} 
%   \item DECLARATIVE SPECIFICATION
%   \begin{itemize}
%     \item Hoare logics, separation logic 
%     \item temporal logics (LTL, CTL, ATL)
%     \item partial correctness vs. termination, safety vs. liveness 
%   \end{itemize}
% \end{enumerate}

% *: Operational.

\section{Math}



\subsection{Function and Sets}
We introduce basic properties of functions and how they are used to evaluate the sizes of sets. 

\begin{definition}
  A real number $b \in \RN$ is a \emph{bound} of a 
  function $f\colon \RN\to \RN$ from $\RN$ to $\RN$, if for all $x$ in 
  $\RN$, we have $f(x) \leq b$. 
\end{definition}


\begin{definition}
  Given two functions $f\colon  \RN\to \RN$ and $g\colon  \RN\to \RN$ from $\RN$ to $\RN$, their \emph{sum} is the function $f + g$ such 
  that for all $x$ in $\RN$, we have $(f+g) (x) = f(x) + g(x)$.
\end{definition} 


\begin{theorem}
  For all functions $f\colon  \RN\to \RN$ and $g\colon  \RN\to \RN$, if $f$ and
  $g$ are bounded, then $f+g$ is bounded. 
\end{theorem}
\begin{proof}
This prove is explicitly presented step by step.
  \begin{enumerate}
    \item Consider arbitrary functions $\hat{f}$ and $\hat{g}$ from 
    $\RN$ to $\RN$.
    \item Assume $\hat{f}$ and $\hat{g}$ are bounded. 
    \item Show that $\hat{f} + \hat{g}$ is bounded.
    \item (2 $\rightarrow$) Let $\hat{a}$ be a bound for $\hat{f}$, 
    and $\hat{b}$ be a bound for $\hat{g}$.
    \item We show that $\hat{a} + \hat{b}$ is a bound for 
    $\hat{f} + \hat{g}$.
    \item Consider an arbitrary real $\hat{x}$.
    \item Show $(\hat{f}+\hat{g})(x) \leq \hat{a} + \hat{b}$.
    \item (Definition of sum) $(\hat{f}+\hat{g})(\hat{x}) 
    = \hat{f}(\hat{x}) + \hat{g}(\hat{x})$.
    \item (Definition of bound) $\hat{f}(\hat{x}) \leq \hat{a}$ and 
    $\hat{g}(\hat{x}) \leq \hat{b}$.
    \item The rest follows from ``arithmetic''. 
  \end{enumerate}
\end{proof}

\begin{definition}
  Two sets $A$ and $B$ are \emph{equipollent} (``have the same 
  size''), if there is a bijection from $A$ to $B$.
\end{definition}

\begin{definition}
  A function $f$ from $A$ to $B$ is: 
  (i) \emph{one-to-one} if for all $x$ and $y$ in $A$, if $x \neq y$, then $f(x) \neq f(y)$;
  (ii) \emph{onto} if for all $z$ in $B$, there exists $x$ 
    in $A$ such that $f(x) = z$;
(iii) \emph{bijective} if $f$ is one-to-one and onto.
\end{definition}

\subsection{Style Guide}
We provide an informal style guide for writing mathematical proofs. 

\begin{center}
    \begin{tabular}{l|l|c}
     Goal & Knowledge & Outermost symbol \\
    \hline 
     \pbox{20cm}{Show for all $x$, $G(x)$. \\ Consider arbitrary $\hat{x}$.\\ Show $G(\hat{x})$} & \pbox{20cm}{We know for all $x$, $K(x)$ \\ In particular we know $K(\hat{t})$ for constant $\hat{t}$} & $\forall$ \\ 
    \hline 
     \pbox{20cm}{Show: exists $x$ s.t. $G(x)$. \\ We show $G(\hat{t})$} & \pbox{20cm}{We know exists $x$ s.t. $K(x)$ \\ Let $\hat{x}$ be s.t. $K(x)$} & $\exists$ \\ 
     \hline 
     \pbox{20cm}{Show $G_1$ iff $G_2$ \\ 1. Show if $G_1$ then $G_2$\\ 2. Show if $G_2$ then $G_1$} & \pbox{20cm}{We know $K_1$ iff $K_2$\\ In particular we know if $K_1$ then $K_2$\\ and if $K_2$ then $K_1$} & $\iff$ \\ 
     \hline 
     \pbox{20cm}{Show if $G_1$ then $G_2$ \\ Assume $G_1$\\ Show $G_2$} & \pbox{20cm}{We know if $K_1$ then $K_2$\\ 1. To show $K_2$ it suffices to show $K_2$\\ 2. Know $K_1$, Also know $K_2$} & $\rightarrow$ \\ 
     \hline 
     \pbox{20cm}{Show $G_1$ and $G_2$\\ 1. Show $G_1$\\ 2. Show $G_2$} & \pbox{20cm}{Know $K_1$ and $K_2$\\ 1. Also Know $K_1$ \\ 2. Also Know $K_2$} & $\wedge$ \\ 
     \hline 
     \pbox{20cm}{Show $G_1$ or $G_2$\\ 1. Assume $\neg G_1$, show $G_2$\\ 2. Assume $\neg G_2$, show $G_1$} & \pbox{20cm}{We know $K_1$ or $K_2$. Show $G$.\\
     1. Assume $K_1$, Show $G$\\ 2. Assume $K_2$, Show $G$ \\ Case split $\uparrow$} & $\vee$ \\ 
     \hline 
     \multicolumn{2}{c|}{\pbox{20cm}{Move Negation Inside, as far as possible}} &  $\neg$ \\ 
     \hline 
\end{tabular}
\end{center}



% \begin{center}
% \begin{tabular}{
%   p{0.35\linewidth} | p{0.35\linewidth} | p{0.2\linewidth}}
%   Goals & Knowledge & Outermost symbol \\
%   \hline
%   \begin{itemize}[leftmargin=*]
%     \item[] Show for all $x$, $G(x)$.
%     \item[] Consider arbitrary $\hat{x}$.
%     \item[] Show $G(\hat{x})$. 
%   \end{itemize} & \begin{itemize}[leftmargin=*]
%     \item[] We know for all $x$, $K(x)$.
%     \item[] In particular, we know $K(\hat{t})$.
%     \item[] $\hat{t}$: term containing only constants. 
%   \end{itemize} & \begin{itemize}
%     \item[] \Large $\forall$
%   \end{itemize} \\ 
%   \hline 
%   \begin{itemize}[leftmargin=*]
%     \item[] Show there exists $x$ s.t. $G(x)$.
%     \item[] We show that $G(\hat{t})$.
%     \item[] $\hat{t}$: term containing only constants.
%   \end{itemize} & \begin{itemize}[leftmargin=*]
%     \item[] We know there exists $x$ s.t. $K(x)$.
%     \item[] Let $\hat{x}$ be s.t. $K(\hat{x})$.
%   \end{itemize} & \begin{itemize}
%     \item[] \Large $\exists$
%   \end{itemize}
% \end{tabular}
% \end{center}

\begin{theorem}[Schr\"{o}der-Bernstein theorem]
    \label{thrm:class1:bernstein}
    If there is a one-to-one function on a set $A$ to a sub-
    set of a set $B$ and there is also a one-to-one function on $B$ to a sub-
    set of $A$, then $A$ and $B$ are equipollent.
\end{theorem}

\begin{exercise}
    Correct the following incorrect proof of Theorem \ref{thrm:class1:bernstein} \emph{in the presented style}:
    \begin{quote}
Suppose that $f$ is a one-to-one map of $A$ into $B$ and $g$ is
one-to-one on $B$ to $A$. It may be supposed that $A$ and $B$ are disjoint. The proof of the theorem is accomplished by decomposing $A$ and $B$ into classes which are most easily described in terms of parthenogenesis. A point $x$ (of either $A$ or $B$) is an ancestor of a point $y$ iff $y$ can be obtained from $x$ by successive application of $f$ and $g$ (or $g$ and $f$). Now decompose $A$ into three sets:
let $A_e$ consist of all points of $A$ which have an even number of ancestors, let $A_o$ consist of points which have an odd number of ancestors, and let $A_{\infty}$ consist of points with infinitely many ancestors. Decompose $B$ similarly and observe: f maps $A_e$ onto
$B_o$ and $A_{\infty}$ onto  $B_{\infty}$, and $g^{-1}$ maps $A_o$ onto $B_e$. Hence the function which agrees with $f$ on $A_e\cup A_{\infty}$, and agrees with $g^{-1}$ on $A_o$ is a one-to-one map of $A$ onto $B$.
    \end{quote}
\end{exercise}
