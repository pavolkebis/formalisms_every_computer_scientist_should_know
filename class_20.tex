% \documentclass{article}
% \usepackage{graphicx} % Required for inserting images


% \renewcommand{\equsim}{\approx_{sim}}
% \providecommand{\equbis}{\approx_{bis}}


% 

%%%%%% Basic Math
\newcommand{\comment}[1]{\textcolor{red}{#1}}

\newcommand{\notimplies}{%
  \mathrel{{\ooalign{\hidewidth$\not\phantom{=}$\hidewidth\cr$\implies$}}}}


\newcommand{\RN}{\mathbb{R}}
\newcommand{\QN}{\mathbb{Q}}
\newcommand{\NN}{\mathbb{N}}
\newcommand{\BN}{\mathbb{B}}
\newcommand{\ZN}{\mathbb{Z}}
\newcommand{\power}{\mathcal{P}}

\newcommand{\true}{\mathbf{true}}
\newcommand{\false}{\mathbf{false}}


\newcommand{\simplies}{\Rightarrow}
\newcommand{\simplied}{\Leftarrow}
\newcommand{\siff}{\Leftrightarrow}
\newcommand{\emptystring}{\epsilon}
\newcommand{\gfp}{\mathrm{gfp}}
\newcommand{\lfp}{\mathrm{lfp}}
\newcommand{\bottom}{\perp}

\newcommand{\valfoo}{\nu}
\newcommand{\eval}[1]{\llbracket #1 \rrbracket}
\newcommand{\interp}{\mathcal{I}}

\newcommand{\sat}{\mathbf{sat}}
\newcommand{\unsat}{\mathbf{unsat}}
%%%%%% Merge Sort

\newcommand{\mergesort}{\mathtt{mergesort}}
\newcommand{\merges}{\mathtt{merge}}
\newcommand{\odd}{\mathtt{odd}}
\newcommand{\even}{\mathtt{even}}
\newcommand{\sorted}{\mathtt{sorted}}
\newcommand{\reverseA}{\mathtt{reverse}}
\newcommand{\reverseB}{\mathtt{reverse2}}

\newcommand{\parent}{\mathtt{parent}}
\newcommand{\monkey}{\mathtt{monkey}}
\newcommand{\human}{\mathtt{human}}


\newcommand{\proofs}{\vdash}


\newcommand{\hilbert}{\mathcal{H}}
% \newcommand{\nj}{\mathcal{N}\mkern-9mu\mathcal{J}}
% \newcommand{\nk}{\mathcal{N}\mkern-6mu\mathcal{K}}
% \newcommand{\lk}{\mathcal{L}\mkern-6mu\mathcal{K}}
\newcommand{\nj}{\mathcal{NJ}}
\newcommand{\nk}{\mathcal{NK}}
\newcommand{\lk}{\mathcal{LK}}
 
\newcommand{\kripke}{m}
\newcommand{\krel}{\preceq}
\newcommand{\kinit}{w_0}
\newcommand{\kworld}{W}


\newcommand{\lif}{\mathrm{if}}
% \newcommand{\ln}{\underline{if}}


\newcommand{\tuple}[1]{\langle #1 \rangle}

\newcommand{\sif}[3]{\mathrm{if}\;#1\;\mathrm{then}\;#2\;\mathrm{else}\;#3}
\newcommand{\swhile}[2]{\mathrm{while}\;#1\;\mathrm{do}\;#2}
\newcommand{\siffi}[1]{\mathrm{if}\;#1\;\mathrm{fi}}
\newcommand{\sdo}[1]{\mathrm{do}\;#1\;\mathrm{od}}
\newcommand{\sskip}{\mathtt{skip}}
\newcommand{\sassert}{\mathtt{assert}}
\newcommand{\sfail}{\mathtt{fail}}
\newcommand{\sassume}{\mathtt{assume}}
\newcommand{\sabort}{\mathtt{abort}}
\newcommand{\ssc}{\mathtt{c}}
\newcommand{\ssb}{\mathtt{b}}
\newcommand{\sgc}{\mathtt{gc}}

\newcommand{\simp}[1]{\mathtt{#1}}
\newcommand{\sop}[2]{\langle \simp{#1},~#2 \rangle}
\newcommand{\sde}[1]{\llbracket \simp{#1} \rrbracket}
\newcommand{\sdefun}[1]{\lambda \sigma.~ #1}

\newcommand{\imp}[1]{\texttt{#1}}
\newcommand{\opsem}[2]{\langle \imp{#1},~#2 \rangle}
\newcommand{\desem}[1]{\llbracket \imp{#1} \rrbracket}
\newcommand{\defun}[1]{\lambda \sigma.~ #1}



\newcommand{\vc}{\mathrm{vc}}
\newcommand{\wlp}{\mathrm{wlp}}
\newcommand{\wsp}{\mathrm{wp}}



\newcommand{\ccs}{\mathcal{P}}

\newcommand{\coin}{\;\mathrm{coin}\,}
\newcommand{\discof}{\;\mathrm{discof}\,}
\newcommand{\reqcof}{\;\mathrm{reqcof}\,}
\newcommand{\distea}{\;\mathrm{distea}\,}
\newcommand{\reqtea}{\;\mathrm{reqtea}\,}


\newcommand{\inc}{\;\mathrm{inc}\,}
\newcommand{\dec}{\;\mathrm{dec}\,}
\newcommand{\zero}{\;\mathrm{zero}\,}
\newcommand{\Counter}{\;\mathrm{Counter}\,}
\newcommand{\semicounter}{\;\mathrm{Semicounter}\,}



\newcommand{\equiso}{\approx_{\mathrm{iso}}}
\newcommand{\equbis}{\approx_{\mathrm{bis}}}
\newcommand{\equsim}{\approx_{\mathrm{sim}}}
\newcommand{\equlin}{\approx_{\mathrm{lin}}}

\newcommand{\lequsim}{\preceq_{\mathrm{sim}}}
\newcommand{\gequsim}{\succeq_{\mathrm{sim}}}

\providecommand{\T}{T}
\DeclareMathOperator{\pre}{pre}
\DeclareMathOperator{\post}{post}
\DeclareMathOperator{\HML}{HML}
\DeclareMathOperator{\HMLlin}{HML_{\mathrm{lin}}}



\newcommand{\bev}[1]{\langle #1 \rangle}
\newcommand{\bal}[1]{[ #1 ]}

\newcommand{\lev}{\lozenge}
\newcommand{\lal}{\square}
\newcommand{\lne}{\bigcirc}
\newcommand{\lun}{\,\mathcal{U}\,}

\newcommand{\lmu}{\mu\,}
\newcommand{\lnu}{\nu\,}

\newcommand{\algo}{\mathcal{A}}


\newcommand{\tatom}{\mathrm{atom}}
\newcommand{\tcar}{\mathrm{car}}
\newcommand{\tcdr}{\mathrm{cdr}}
\newcommand{\tcons}{\mathrm{cons}}

\newcommand{\tread}{\mathrm{read}}
\newcommand{\twrite}{\mathrm{write}}



% \commenter{ruichen}{violet}




% \section{$\equsim$ and $\equbis$}


\begin{definition}
    We define the predecessor and successor of a subset $\sigma\subseteq S$ for some actions $a\in A$ respectively as 
    \begin{align*}
         \pre_a(\sigma) \coloneqq \{ s \mid  \exists s' \in \sigma.\ s \xrightarrow{a} s' \} \\
          \post_a(\sigma) \coloneqq \{ s' \mid \exists s \in \sigma.\ s \xrightarrow{a} s' \} 
    \end{align*}
\end{definition}

\begin{remark}
    The set $\pre_a(\sigma)$ is the set of states that can reach $\sigma$ by action $a$ and the set $\post_a(\sigma)$ is the set of states that can be reached from $\sigma$ by action $a$.
    \begin{center}
         \begin{tikzpicture}
    % Draw the three intersecting circles
    \draw (0,0) circle (1cm) node[below] {\small $ \pre_a(\sigma)$};  % First circle
    \draw (1.5,0) circle (0.8cm) node[below] {\small $\sigma$};  % Center circle
    \draw (3,0) circle (1 cm) node[below] {\small $\post_a(\sigma)$};  % Third circle
    
    % Draw the arrows
    % Arrow from the middle circle to the first circle
    \draw[->, bend left] (-0.5, -0.1) to node[midway, above] {\small$a$}  (1.2, 0.3) ;
    % Arrow from the middle circle to the third circle
    \draw[->,bend left] (1.6, 0.3) to  node[midway, above] {\small$a$}  (3.5, 0.2);
\end{tikzpicture}
    \end{center}
\end{remark}

\begin{theorem}
    \begin{align*}
        s \lequsim t &\iff \exists R \subseteq S \times S.\ (s,t) \in R \simplies \forall a\ \forall s' \in \post_a(s)\ \exists t' \in post_a(t).  (s', t') \in R  \\
          s \equsim t &\iff s \lequsim t \land  t \lequsim s  \\
           s \equbis t &\iff \exists R \subseteq S \times S.\ (s,t) \in R \simplies \forall a\left[
            \begin{array}{l}
               ( \forall s' \in \post_a(s)\ \exists t' \in \post_a(t). (s', t') \in R )\; \land \\
                ( \forall t' \in \post_a(t)\ \exists s' \in \post_a(s). (s', t') \in R )
            \end{array}
        \right]
    \end{align*}
\end{theorem}


\subsection{Characterization}


\begin{theorem}[Game theoretic characterization of $\lequsim$]
The game theoretic characterization  $\lequsim$ is given be the following game.
\begin{enumerate}
    \item Put red token on $s$, and blue token on $t$.
    \item Repeat:
        \begin{enumerate}
            \item Player $1$ moves the \emph{red} token to a successor state.
            \item Player $2$ moves the \emph{blue} button to a successor state, matching the action of the player $1$ move.
        \end{enumerate}
\end{enumerate}
If in the following game player $1$ has a strategy s.t. at some point, player $2$ cannot match player $1$'s move, then $s \not\lequsim t$, otherwise $s \lequsim t$.
\end{theorem}


\begin{theorem}[Game theoretic characterization of $\equbis$]
The game theoretic characterization  $\lequsim$ is given be the following game.
\begin{enumerate}
    \item Put red token on $s$, and blue token on $t$.
    \item Repeat:
        \begin{enumerate}
            \item Player $1$ chooses either the red or blue token.
            \item Player $1$ moves the \emph{chosen} token to a successor state.
            \item Player $2$ moves the \emph{other} button to a successor state, matching the action of the player $1$ move.
        \end{enumerate}
\end{enumerate}
If in the following game player $1$ has a strategy s.t. at some point, player $2$ cannot match player $1$'s move, then $s \not\equbis t$, otherwise $s \equbis t$.
\end{theorem}



% \begin{definition}[Game theoretic of $\equsim$]
% The game theoretic characterization of $\lequsim$ and $\equbis$ is the following.
% For $\lequsim$, if in the following game player $1$ has a strategy s.t. at some point, player $2$ cannot match player $1$'s move, then $s \not\lequsim t$, otherwise $s \lequsim t$.
% \begin{enumerate}
%     \item Put red token on $s$, and blue token on $t$.
%     \item Repeat:
%         \begin{enumerate}
%             \item Player $1$ moves the \emph{red} token to a successor state.
%             \item Player $2$ moves the \emph{blue} button to a successor state, matching the action of the player $1$ move.
%         \end{enumerate}
% \end{enumerate}
% For $\equbis$ the same holds for the following game.
% \begin{enumerate}
%     \item Put red token on $s$, and blue token on $t$.
%     \item Repeat:
%         \begin{enumerate}
%             \item Player $1$ chooses either the red or blue token.
%             \item Player $1$ moves the \emph{chosen} token to a successor state.
%             \item Player $2$ moves the \emph{other} button to a successor state, matching the action of the player $1$ move.
%         \end{enumerate}
% \end{enumerate}
% \end{definition}


% \begin{definition}[Fixpoint of $\equsim$]
% The fixpoint characterization of $\equsim$ and $\equbis$ is the following.
% For 
% For $\lequsim$, if in the following game player $1$ has a strategy s.t. at some point, player $2$ cannot match player $1$'s move, then $s \not\lequsim t$, otherwise $s \lequsim t$.
% \begin{enumerate}
%     \item Put red token on $s$, and blue token on $t$.
%     \item Repeat:
%         \begin{enumerate}
%             \item Player $1$ moves the \emph{red} token to a successor state.
%             \item Player $2$ moves the \emph{blue} button to a successor state, matching the action of the player $1$ move.
%         \end{enumerate}
% \end{enumerate}
% For $\equbis$ the same holds for the following game.
% \begin{enumerate}
%     \item Put red token on $s$, and blue token on $t$.
%     \item Repeat:
%         \begin{enumerate}
%             \item Player $1$ chooses either the red or blue token.
%             \item Player $1$ moves the \emph{chosen} token to a successor state.
%             \item Player $2$ moves the \emph{other} button to a successor state, matching the action of the player $1$ move.
%         \end{enumerate}
% \end{enumerate}
% \end{definition}

% \begin{enumerate}
%     \item Put red token on $s$, and blue token on $t$.
%     \item Repeat:
%         \begin{enumerate}
%             \item ($\equbis$) Player $1$ chooses either red or blue, and moves the chosen token to a successor state. ($\equsim$) Player $1$ moves the red token to a successor state.
%             \item ($\equbis$) Player $2$ moves the other token to a successor state, matching the action of the player $1$ move. ($\equsim$) Player $2$ moves the blue button to a successor state, matching the action of the player $1$ move.
%         \end{enumerate}
% \end{enumerate}




\begin{algorithm}[htp]
\caption{$\algo_{\equbis}$}
\label{alg:bis}
\begin{algorithmic}
    \State $X \gets \{ S \}$
    \State $X' \gets \{ \emptyset\}$
    \While{$X'\not=X$} 
        \State $X' \gets X$
        \State $X \gets X \cup \{ \sigma \cup \pre_a(\sigma') | \sigma, \sigma' \in X,\ a \in A \} \cup \{ \sigma \setminus \pre_a(\sigma') | \sigma, \sigma' \in X,\ a \in A \}$
    \EndWhile
    \If{$\forall \sigma\in X.\; s,t \in \sigma \lor s,t \notin \sigma$ }
        \State \Return $\true$
    \EndIf
    \State \Return $\false$
\end{algorithmic}
\end{algorithm}



\begin{theorem}[Fixpoint characterization of $\equbis$]
        Algorithm $\algo_{\equbis}$ provides a fixed point characterization of $\equbis$, i.e., $\forall s, t\in S$ we have that $s \equbis t$ iff  $\algo_{\equbis}$ terminates with $\true$.
\end{theorem}

\begin{algorithm}[htp]
\caption{$\algo_{\equsim}$}
\label{alg:bis}
\begin{algorithmic}
    \State $X \gets \{ S \}$
    \State $X' \gets \{ \emptyset\}$
    \While{$X'\not=X$} 
        \State $X' \gets X$
        \State $X \gets X \cup \{ \sigma \cup \pre_a(\sigma') | \sigma, \sigma' \in X,\ a \in A \}$
    \EndWhile
    \If{$\forall \sigma\in X.\; s,t \in \sigma \lor s,t \notin \sigma$ }
        \State \Return $\true$
    \EndIf
    \State \Return $\false$
\end{algorithmic}
\end{algorithm}



\begin{theorem}[Fixpoint characterization of $\equsim$]
        Algorithm $\algo_{\equsim}$ provides a fixed point characterization of $\equsim$, i.e., $\forall s, t\in S$ we have that $s \equsim t$ iff  $\algo_{\equsim}$ terminates with $\true$.
\end{theorem}

\begin{definition}[Hennessy-Milner Logic ($\HML$) Syntax]
    The syntax of Hennessy-Milner Logic ($\HML$) is given by the following grammar.
    \begin{align*}
        \varphi \Coloneqq\true  \; \mid \;  \false \; \mid \;  \varphi \land \varphi \; \mid \;  \varphi \lor \varphi \; \mid \;  \bev{a} \varphi  ,\;  a \in A \; \mid \;  \bal{a} \varphi,\;  a \in A
    \end{align*}
    \end{definition}


\begin{definition}[Hennessy-Milner Logic ($\HML$) Semantics]
    The semantics of Hennessy-Milner Logic ($\HML$) is defined as follows. 
    Let $T\coloneqq (S, \to)$ be a labeled transition system, let $\varphi$ a $\HML$ formula, let $s\in S$, then we define $s\models \varphi$ inductively:
    \begin{itemize}[noitemsep]
        \item if $ \varphi = \varphi_1 \land \varphi_2 $, then $ s \models_T \varphi_1 \land \varphi_2$ iff $s \models_T \varphi_1$ and  $\models_T \varphi_2$;
        \item if $ \varphi = \varphi_1 \lor \varphi_2 $, then $ s \models_T \varphi_1 \lor \varphi_2$ iff $s \models_T \varphi_1$ and  $\models_T \varphi_2$;
      \item if $ \varphi = \bev{a}\varphi $, then $ s \models_T\bev{a}\varphi $ iff $\exists s' \in \post_a(s).\ s' \models_T \varphi$ \\
      (equivalent to $\exists s' .\; s \xrightarrow{a} s' \land  \varphi$); 
    \item if $ \varphi = \bal{a}\varphi $, then $ s \models_T\bal{a}\varphi $ iff $\forall s' \in \post_a(s).\ s' \models_T \varphi$ \\
      (equivalent to $\forall s' .\; s \xrightarrow{a} s' \simplies  \varphi$); 
    \end{itemize}
    Moreover, we define $\eval{\varphi}_T\subseteq S$ s.t. $\eval{\true}_T\coloneqq S$, $\eval{\false}_T \coloneqq  \emptyset$, $\eval{\varphi_1 \land \varphi_2}_T \coloneqq  \eval{\varphi_1 }_T \cap \eval{\varphi_2}_T$, and $\eval{\varphi_1 \lor \varphi_2}_T \coloneqq  \eval{\varphi_1 }_T \cup \eval{\varphi_2}_T$.
\end{definition}

\begin{definition}[$\HML$ Fragments]
     The logic $\HML^\exists$ is the $\HML$ without $\bal{a}$. The logic $\HML ^\forall$ is the $\HML$ without $\bev{a}$. The logic $\HML_\neg$ adds negation, i.e., for any $\HML$ formula we add $\neg \varphi$.
\end{definition}

\begin{lemma}
    In $\HML_{\neg}$ the following equivalences hold
    \begin{align*}
        \neg \true &\equiv \false \\
        \neg (\varphi_1 \land \varphi_2) &\equiv \neg \varphi_1 \lor \neg\varphi_2 \\
        \neg \bev{a}\varphi &\equiv \bal{a}\neg \varphi \\
        \neg \bal{a}\varphi &\equiv \bev{a}\neg \varphi \\
    \end{align*}
\end{lemma}

\begin{theorem}[Logical Characterization of $\equbis$]
    The $\equbis$ is logically characterized by $\HML$, i.e.,  $s \equbis t$ iff $\forall \varphi \in \HML$, either $s,t \models \varphi$ or $s,t \not\models \varphi$. 
\end{theorem}

\begin{theorem}[Logical Characterization of $\equsim$]
    The $\equsim$ is logically characterized by $\HML^{\exists}$ (or $\HML^{\forall}$), i.e.,
    $s \equsim t$ iff $\forall \varphi \in \HML ^\exists$, either $s,t \models \varphi$ or $s,t \not\models \varphi$. (Alternatively,  $s \equsim t$  iff $\forall \varphi \in \HML ^\forall$, either $s,t \models \varphi$ or $s,t \not\models \varphi$). 
\end{theorem}


\subsection{State equivalence language (or trace) equivalence}


\begin{definition}
    Let $T= (S, \to)$ be an LTS, let $s\in S$, then we define
    \begin{align*}
        L_T(s) \coloneqq \{ a_1a_2\cdots a_n | \exists s_0 \cdots s_n.\ s_0 \xrightarrow{a_1}  s_1  \xrightarrow{a_2} \cdots \xrightarrow{a_n} s_n \land s_0 = s \}
    \end{align*}
\end{definition}


\begin{lemma}
    $L_T(s)$ contains $\varepsilon$ and is prefix-closed.
\end{lemma}


\begin{definition}[Linear Equivalence]
    $s_1 \equlin s_2$ iff $L_T(s_1) = L_T(s_2)$.
\end{definition}



\begin{remark}
    $s_1 \equsim s_2$ implies $s_1 \equlin s_2$ (but not the other direction).
    The problem of $s_1 \stackrel{?}{\equlin} s_2$ reduces to $L_T(s_1) \stackrel{?}{=} L_T(s_2)$. 
    The latter is known as the language inclusion problem and corresponds to the equivalence problem for finite non-deterministic automata (NFA).  
\end{remark}

\begin{remark}
To decide whether $L_{\T_1}(s) = L_{\T_2}(s)$, we decide whether $L_{\T_1}(s) \subseteq L_{\T_2}(s)$ and $L_{\T_2}(s) \subseteq L_{\T_1}(s)$. However, for two languages $L_1$ and $L_2$ accepted by NFAs we have $L_1 \subseteq L_2$ iff $L_1 \cap A^* \setminus L_2 = \empty$. Hence, the language inclusion problem is $\mathrm{PSPACE}$-hard.
To elaborate while the intersection of two languages is polynomial and the emptiness check for NFAs is linear, the step $A^* \setminus L_2 $ requires the determinization of an NFA which is exponential. 
\end{remark}

\begin{example}
    Below an example of the determinization of a NFA.
        \begin{center}
    \begin{tikzpicture}[->, >=Stealth, node distance=2cm, thick, 
    main/.style = {draw, ellipse}, 
    dot/.style = {circle, fill, inner sep=1.5pt}]
    % Nodes

    \node[dot] (4) at (0, 0) {};
    \node[dot] (5) at (-1, -1) {};
    \node[dot] (6) at (1, -1) {};
    \node[dot] (7) at (-1, -2) {};
    \node[dot] (8) at (1, -2) {};


    \node[dot] (11) at (5, 0) {};
    \node[dot] (12) at (5, -2) {};
    \node[dot] (13) at (7, 0) {};
    \node[dot] (14) at (7, -2) {};
    \node[dot] (15) at (9, -2) {};
    \node[dot] (16) at (9, 0) {};

    % Edges
  

    \draw  (4) to node[midway, below] {\small$a$} (5) ;
    \draw  (5) to node[midway, below] {\small$b$} (7) ;
    \draw  (4) to node[midway, below] {\small$a$} (6) ;
    \draw  (6) to node[midway, below] {\small$c$} (8) ;


    \draw  (13) to node[midway, right] {\small$a$} (14) ;
    \draw  (14) to node[midway, below] {\small$b$} (12) ;
    \draw  (12) to node[midway, left] {\small$a,b,c$} (11) ;
    \draw  (13) to node[midway, below] {\small$b,c$} (11) ;
    \draw  (14) to node[midway, below] {\small$a$} (11) ;
    \draw  (14) to node[midway, below] {\small$c$} (15) ;
    \draw  (15) to node[midway, right] {\small$a,b,c$} (16) ;
    
    

\end{tikzpicture}
 \end{center}
    
\end{example}


\begin{definition}[$\HMLlin$]
    The fragment $\HMLlin^\exists$ of $\HML$ is characterized by the grammar
    \begin{align*}
        \varphi \Coloneqq true \; \mid \; \bev{a} \varphi        
    \end{align*}
    The fragment $\HML _{lin}^\forall$ is defined analogously. 
\end{definition}


\begin{example}
    Below an example of comparing bisimulation and linear equivalence
    \begin{center}
    \begin{tikzpicture}[->, >=Stealth, node distance=2cm, thick, 
    main/.style = {draw, ellipse}, 
    dot/.style = {circle, fill, inner sep=1.5pt}]
    % Nodes

    \node[dot] (1) at (0, 0) {};
    \node[dot] (2) at (0, -1) {};
    \node[dot] (3) at (0, -2) {};


    \node (b) at (1,-1) {\large$\not\equlin$};
  
    
    \node[dot] (4) at (3, 0) {};
    \node[dot] (5) at (2, -1) {};
    \node[dot] (6) at (4, -1) {};
    \node[dot] (7) at (2, -2) {};
    \node[dot] (8) at (4, -2) {};

    \node (b) at (5,-0.5) {\large$\equlin$};
    \node (b) at (5,-1.5) {\large$\not\equbis$};
        
    \node[dot] (9) at (7, 0) {};
    \node[dot] (10) at (7, -1) {};
    \node[dot] (11) at (6, -2) {};
    \node[dot] (12) at (8, -2) {};
   

    % Edges
  
    \draw  (1) to node[midway, right] {\small$a$} (2) ;
    \draw  (2) to node[midway, right] {\small$b$} (3) ;
    
    \draw  (4) to node[midway, right] {\small$a$} (5) ;
    \draw  (5) to node[midway, right] {\small$b$} (7) ;

    \draw  (4) to node[midway, right] {\small$a$} (6) ;
    \draw  (6) to node[midway, right] {\small$c$} (8) ;

    \draw  (9) to node[midway, right] {\small$a$} (10) ;
    \draw  (10) to node[midway, right] {\small$b$} (11) ;
    \draw  (10) to node[midway, right] {\small$c$} (12) ;
    
    

\end{tikzpicture}
 \end{center}
\end{example}

% \begin{theorem}
%     $s \equbis t$ iff after Algorithm~\ref{alg:bis} terminates, either $s,t \in \sigma$, or $s,t \notin \sigma$. $s \equsim t$ iff after Algorithm~\ref{alg:sim} terminates, either $s,t \in \sigma$, or $s,t \notin \sigma$. 
% \end{theorem}



% \subsection{Logical charcterization}

% \begin{definition}[Hennessy-Milner Logic ($\HML$)]
%     Syntax: $\varphi = true\ |\ false\ |\ \varphi \land \varphi\ |\ \varphi \lor \varphi\ |\ \bev{a} \varphi\ |\ \bal{a} \varphi$, where $a \in A$.
    
%     Semantics: 
%     \begin{itemize}
%         \item $\T = (S, \to)$
%         \item $s \in S$
%         \item $\llbracket \varphi \rrbracket _\T \subseteq S$
%         \item $S \vDash _\T \varphi_1 \land \varphi_2$ iff $S \vDash_\T \varphi_1$ and $S \vDash _\T \varphi_2$
%         \item $S \vDash _\T \varphi_1 \lor \varphi_2$ iff $S \vDash_\T \varphi_1$ or $S \vDash _\T \varphi_2$
%         \item $S \vDash _\T \bev{a} \varphi$ iff $\exists s' \in \post_a(s).\ s' \vDash_\T \varphi$
%         \item $S \vDash _\T [ {a} ] \varphi$ iff $\forall s' \in \post_a(s).\ s' \vDash_\T \varphi$
%         \item $\llbracket true \rrbracket _\T = S$
%         \item $\llbracket false \rrbracket _\T = \varphi$
%         \item $\llbracket \varphi_1 \land \varphi_2 \rrbracket _\T = \llbracket \varphi_1 \rrbracket _\T \cap \llbracket \varphi_2 \rrbracket _\T$
%         \item $\llbracket \varphi_1 \lor \varphi_2 \rrbracket _\T = \llbracket \varphi_1 \rrbracket _\T \cup \llbracket \varphi_2 \rrbracket _\T$
%     \end{itemize}
% \end{definition}

% \begin{definition}
   
% \end{definition}

% \begin{theorem}
%     $s \equbis t$ iff $\forall \varphi \in \HML$, either $s,t \vDash \varphi$ or $s,t \nvDash \varphi$. $s \equsim t$ iff $\forall \varphi \in \HML ^\exists$, either $s,t \vDash \varphi$ or $s,t \nvDash \varphi$, iff $\forall \varphi \in \HML ^\forall$, either $s,t \vDash \varphi$ or $s,t \nvDash \varphi$. 
% \end{theorem}

% \section{State equivalence language (or trace) equivalence}

% Semantics:
% \begin{itemize}
%     \item $\T = (S, \to)$
%     \item $s \in S$
%     \item $L_\T = \{ a_1a_2\cdots a_n | \exists s_0 \cdots s_n.\ s_0 \stackrel {a_1} \to s_1 \stackrel {a_2} \to \cdots \stackrel {a_n} \to s_n \land s_0 = s \}$
% \end{itemize}

% \begin{lemma}
%     $L_\T(s)$ contains $\varepsilon$ and is prefix-closed.
% \end{lemma}

% \begin{definition}
%     $s_1 \approx_{lin} s_2$ iff $L_\T (s_1) = L_\T (s_2)$.
% \end{definition}

% \begin{remark}
%     $s_1 \equsim s_2$ implies $s_1 \approx_{lin} s_2$. (but not the other direction)
% \end{remark}

% To decide whether $L_{\T_1}(s) = L_{\T_2}(s)$, we decide whether $L_{\T_1}(s) \subseteq L_{\T_2}(s)$ and $L_{\T_2}(s) \subseteq L_{\T_1}(s)$. 

% \begin{remark}
%     $L_1 \subseteq L_2$ iff $L_1 \cap A^\star \setminus L_2 = \empty$. In total, this is PSPACE-hard.
% \end{remark}

% \begin{definition}[$\HML _{lin} ^\exists$]
%      $\varphi := true\ |\ \bev{a} \varphi$.
% \end{definition}

%
%% \newpage
% \bibliographystyle{plainnat}
% %\bibliographystyle{myplainnat} %a bit nicer for arxiv, revert to plainnat once bib is polished
% {\small
% \bibliography{reference}
% }


