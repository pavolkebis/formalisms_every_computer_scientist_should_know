\subsection{Encoding $\mathbb{N}$ in $\lambda$ (Church numerals)}



\begin{definition}[$\BN$]
The booleans $\BN$ are defined in $\lambda$-calculus as
\begin{align*}
    \top \coloneqq \lambda x \lambda y.x = \lambda x y. x  \quad \text{and} \quad  \bot \coloneqq \lambda x \lambda y.y = \lambda x y. y.
\end{align*}
Moreover, if is defined as $\lif \coloneqq \lambda b x y. bxy$.
    
\end{definition}


\begin{example}
The expression $\lif \top  s  t$ reduces as follows.
\begin{align*}
\lif \top  s  t  = \quad  &(\lambda bxy.bxy)(\lambda xy.x)st \to (\lambda xy.(\lambda xy.x)xy)st \to (\lambda y.(\lambda xy.x)sy)t \to \\
&(\lambda xy.x)st \to (\lambda y. s)t \to s 
\end{align*}
We can derive $t$ from $if$ $\bot st$ similarly.
\end{example} 



\begin{definition}[$\NN$]
The natural numbers $\NN$ are defined in $\lambda$-calculus as
\begin{align*}
    \underline{0} \coloneqq \lambda fx.x ,\;  
    \underline{1} \coloneqq \lambda fx.fx \; , 
    \underline{2} \coloneqq \lambda fx.f(fx) \;, 
    \underline{n} \coloneqq \lambda fx.\underbrace{f(\dots f }_{{n \text{times}}}x)
\end{align*}
The successor function is defined as $S \coloneqq \lambda nfx. f(nfx)$.
\end{definition}


\begin{example}
The expression $\lif \top  s  t$ reduces as follows.
\begin{align*}
    S\underline{n} =\quad & (\lambda nfx. f(nfx))(\lambda fx .f^nx) \to \lambda fx. f((\lambda fx. f^nx)fx) \to \lambda fx.f((\lambda x. f^n x)x) \to \\
    &\lambda fx. f(f^n x) \to \lambda fx. f^{n+1}x \to n+1 
\end{align*}
\end{example}


\begin{definition}[Fixed point operator]
    The expression $Y = \lambda f.(\lambda x. f(xx))(\lambda x. f(xx))$ is a fixed point operator.
\end{definition}



\begin{theorem}
	$Yt =_{\beta} t(Yt)$  are $\beta$ equivalence. That is, 
there exists $t'$ that both $Yt$ and $t(Yt)$ can be $\beta$-reduced to $t'$.
\end{theorem}
\begin{proof}
    First the left side
    \begin{align*}
        Yt= \quad & (\lambda f.(\lambda x.f(xx))(\lambda x. f(xx)))t \to (\lambda x. t(xx))(\lambda x. t(xx)) \to t((\lambda x. t(xx))(\lambda x. t(xx))).
    \end{align*}
    Second the right side 
	\begin{align*}
        t(Yt)= \quad &  t((\lambda f.(\lambda x.f(xx))(\lambda x. f(xx)))t) \to t((\lambda x. t(xx))(\lambda x. t(xx))).
    \end{align*}
\end{proof}

\begin{example}
	$\mathrm{fact}\; = \lambda fn. \lif \;(\mathrm{iszero } \;n) 1 \mathrm{ else} \;(\mathrm{mult }\; n(f (\mathrm{pred }\; n)))$
\end{example}


\subsection{Combinatory logic}


\begin{definition}[Combinatory logic]
    In combinatory logic is defined with the following combinators
    \begin{align*}
        I \coloneqq \lambda x.x \quad  S \coloneqq \lambda xyz. (xz)(yz) \quad K = \lambda xy.x
    \end{align*}
\end{definition}


\begin{theorem}
Every $\lambda$ term can be built from $I$, $S$, and $K$.
\end{theorem}

\begin{example}
    \begin{align*}
        K((SI)I) =\quad & (\lambda xy.x)(((\lambda xyz. (xz)(yz))(\lambda x.x))(\lambda x.x))\to  \\
        &\lambda y. (((\lambda xyz. (xz)(yz))(\lambda x.x))(\lambda x.x))
        \to \\
        &\lambda y. ((\lambda yz.((\lambda x.x)z)(yz))(\lambda x.x))\to  \\
       & \lambda y \lambda z ((\lambda x.x)z)((\lambda x.x)z) \to \\
       &\lambda yz. zz
    \end{align*}
\end{example}


\begin{exercise}
	This exercise is threefold. (i) Define the $\lambda$-terms of functions $\mathrm{iszero}$, $\mathrm{mult}$, and $\mathrm{pred}$. 
 (ii) Evaluate $\mathrm{fact}(3)$ ($\rightarrow_\beta^\star 6$).
 (iii) Write $\mathrm{fact}$ in terms of $I$, $S$, and $K$.
\end{exercise}

\subsection{Simply typed $\lambda$}



\begin{definition}[Simply typed $\lambda$-calculus]
    To define simply typed $\lambda$-calculus we define:  types, context, terms, typing judgments, and typing rules.
    \begin{itemize}
        \item The \emph{Types} are defined by the grammar
    \begin{align*}
        A \Coloneqq X \; | \; A\to A
    \end{align*}
    with $A \to B$ being a right associates function from $A$ to $B$ (,e.g., $A \to B \to C = A \to (B \to C)$).
    \item  The \emph{context} $\Gamma\coloneqq \{x_1\colon A_1, \dots x_n\colon A_n \}$ provides a type $A_i$ to each $\lambda$-variable $x_i$ e.g. $\Gamma(x_i) = A_i$. Note that $\mathrm{domain}(\Gamma) = \{x_1, ..., x_n\}$.
    \item The terms are defined by the following grammar
    \begin{align*}
       t \Coloneqq x \; | \; t_1t_2 \; | \; \lambda x\colon A.t
    \end{align*}
    \item The \emph{typing judgments} are of the form $\Gamma \proofs t\colon  A$.
    \item There are \emph{typing rules} three typing rules, axiom (ax), application (ap), and abstraction (ab)
    \begin{center}
        \AxiomC{}
        \RightLabel{\small(ax)}
        \UnaryInfC{$\Gamma \proofs x \colon \Gamma(x)$}
        \DisplayProof
        \AxiomC{$\Gamma \proofs s\colon A$}
        \AxiomC{$\Gamma \proofs t\colon A \to B$}
        \RightLabel{\small(ap)}
        \BinaryInfC{$\Gamma \proofs ts\colon  B$}
        \DisplayProof 
    \end{center}
        \begin{center}
        \AxiomC{$\Gamma [x \mapsto A] \proofs t\colon B$}
        \RightLabel{\small(ab)}
        \UnaryInfC{$\Gamma \proofs (\lambda x\colon A.t)\colon A \to B$}
        \DisplayProof
    \end{center}
    \end{itemize}
    
   
\end{definition}

\begin{remark}
   
\end{remark}

\begin{example}
	$(\lambda f\colon  A \to A. \lambda x\colon A. f(fx))\colon  (A \to A) \to A \to A$
\end{example}
\begin{proof}
The proof tree has the following shape
\begin{center}
\small
\begin{prooftree}
        \AxiomC{$\true$}
         \RightLabel{\small(ax)}
        \UnaryInfC{$\{f\colon  A \to A, x\colon A\} \proofs f\colon A \to A$}
    \AxiomC{$\true$}
    \UnaryInfC{$\{f\colon  A \to A, x\colon A\} \proofs f\colon A \to A$}
    \AxiomC{$\true$}
    \UnaryInfC{$\{f\colon  A \to A, x\colon A\} \proofs x\colon A $}
    \BinaryInfC{$\{f\colon  A \to A, x\colon A\} \proofs fx\colon A$}
    \RightLabel{\small(ap)}
    \BinaryInfC{$\{f\colon  A \to A, x\colon A\} \proofs (f(fx))\colon A$}
    \RightLabel{\small(ab)}
    \UnaryInfC{$\{f\colon  A \to A\} \proofs (\lambda x\colon A. f(fx))\colon A \to A$}
    \RightLabel{\small(ab)}
    \UnaryInfC{$\proofs (\lambda f\colon  A \to A. \lambda x\colon A. f(fx))\colon (A \to A) \to A \to A$}
\end{prooftree}
\end{center}

% \begin{prooftree}
%     \AxiomC{True}
%     \AxiomC{True}
%     \RightLabel{\small(ab)}
%     \UnaryInfC{$\{f\colon A \to A, x\colon A\} \proofs f\colon A \to A$}
%     \RightLabel{\small(var)}
%     \UnaryInfC{$\{f\colon A \to A, x\colon A\} \proofs x\colon A$}
%     \RightLabel{\small(app)}
%     \BinaryInfC{$\{f\colon A \to A, x\colon A\} \proofs (fx)\colon A$}
%     \RightLabel{\small(app)}
%     \UnaryInfC{$\{f\colon A \to A, x\colon A\} \proofs (f(fx))\colon A$}
%     \RightLabel{\small(ab)}
%     \UnaryInfC{$\{f\colon A \to A\} \proofs (\lambda x\colon A. f(fx))\colon A \to A$}
%     \RightLabel{\small(ab)}
%     \UnaryInfC{$\proofs (\lambda f\colon A \to A. \lambda x\colon A. f(fx))\colon (A \to A) \to A \to A$}
% \end{prooftree}
% \end{center}
% 	  \[ \infer*
% 	{ \infer*
% 		{ \infer*
% 			{ \infer*
% 				{\infer* {True\\ True}
% 				{True \\ \{f\colon  A \to A, x\colon A\} \proofs f\colon A \to A \\ \{f\colon  A \to A, x\colon A\} \proofs x\colon      A}}
% 			{\{f\colon  A \to A, x\colon A\} \proofs f\colon A \to A \\ \{f\colon  A \to A, x\colon A\} \proofs (fx)\colon A}
% 			}{\{f\colon  A \to A, x\colon A\} \proofs (f(fx)      )\colon A}
% 		}{\{f\colon  A \to A\} \proofs (\lambda x\colon A. f(fx))\colon A \to A}
% 	}{\proofs (\lambda f\colon  A \to A. \lambda x\colon A. f(fx))\colon       (A \to A) \to A \to A}
% 	\]
\end{proof}

\begin{definition}[Problems]
    The main problems are: (i) \emph{type checking}, i.e., given $\Gamma$, $t$, $A$, is $\Gamma \proofs t\colon A$?
    (ii) \emph{type inference}, i.e.,  given $\Gamma$, $t$, is $t$ "typable" and if so, find $A$ such that $\Gamma \proofs t\colon A$?
\end{definition}

\begin{theorem}
    In simply typed $\lambda$, type inference is $O(|t|)$ i.e. linear time.
\end{theorem}

\begin{theorem}[Curry-Howard isomorphism]
    The Curry-Howard isomorphism (also known as correspondence, propositions as types, and proofs as programs) expresses that there is a one to one correspondence between type derivations of $\Gamma \proofs t\colon A$ and proof derivations of $\Gamma \proofs_{\nj} A $. 
\end{theorem}



\begin{definition}[$\beta$-reduction]
    The $\beta$-reduction in simply typed $\lambda$-calculus is the same as for the untyped $\lambda$-calculus, i.e.,
    \begin{align*}
     (\lambda x\colon A.t)s \to_{\beta} t[x\mapsto s]
    \end{align*}
    $(\lambda x\colon A.t)s$ is typable if $\exists B.(t\colon G \land s\colon A)$ (also ($\lambda x\colon A.t\colon  A \to B$))
\end{definition}



\begin{theorem}[Subject reduction theorem]
	If $\Gamma \proofs t\colon A$ and $t \rightarrow_\beta t', then \Gamma \proofs t'\colon A$.
\end{theorem}

\begin{remark}
   Same as cut-elimination in $\nj$, i.e., cut-free proofs is equivalent to typable terms in normal form. 
   \begin{center}
    \AxiomC{$\vdots$}
    \UnaryInfC{$\Gamma, A \proofs B$}
    \UnaryInfC{$\Gamma \proofs A \simplies B$}
    \AxiomC{$\vdots$}
    \UnaryInfC{$\Gamma \proofs A$}
    \BinaryInfC{$\Gamma \proofs B$}
    \DisplayProof
    $\quad \rightsquigarrow \quad $
    \AxiomC{$\vdots$}
    \UnaryInfC{$\Gamma \proofs B$}
    \DisplayProof
\end{center}
\end{remark}


% \[ \infer*
% { \infer*
% 	{ \infer*
% 		{...}{\Gamma, A \proofs B}
% 	}{\Gamma \proofs A \Rightarrow B \\ \Gamma \proofs A}
% }{\Gamma \proofs B}
% \]

\begin{remark}
    Untyped $\lambda$-calculus is confluent and undecidable, i.e., all partial recursive functions are expressible. 
    Simply typed $\lambda$-calculus is confluent and satisfies subject reduction.
    Regarding the expressiveness of simply typed $\lambda$ over church numerals, every typable term corresponds to an extended polynomial function.
\end{remark}

\begin{theorem}
	Every typable term is strongly normalizable.
\end{theorem}

\begin{corollary}
    (i) $Y$ is not typable and  (ii) $=_\beta$ is decidable.
\end{corollary}

