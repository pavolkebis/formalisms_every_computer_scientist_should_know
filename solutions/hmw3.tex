\documentclass{article}


\usepackage{amsthm}
\usepackage{amsmath}
\usepackage{aligned-overset}
\usepackage{graphicx}
\usepackage{stmaryrd}
\usepackage{emptypage}
\usepackage{amssymb}
\usepackage{enumitem}
\usepackage[hidelinks,breaklinks]{hyperref}
\usepackage{mathrsfs}
\theoremstyle{definition}
\newtheorem{definition}{\scshape Definition}
\newtheorem{theorem}[definition]{\scshape Theorem}
\newtheorem{lemma}[definition]{\scshape Lemma}
\newtheorem{observation}[definition]{\scshape Observation}
\newtheorem{corollary}[definition]{\scshape Corollary}
\newtheorem{proposition}[definition]{\scshape Proposition}
\newtheorem{remark}[definition]{\scshape Remark}

\usepackage{bussproofs}
% \input{structure.tex} % Include the file specifying the document structure and custom commands

%----------------------------------------------------------------------------------------
%	ASSIGNMENT INFORMATION
%----------------------------------------------------------------------------------------

% \title{Formalisms for CS: Assignment \#3} % Title of the assignment

% \author{Pavol Kebis\\ \texttt{pavol.kebis@ist.ac.at}} % Author name and email address

% \date{Institute of Science and Technology Austria} % University, school and/or department name(s) and a date

%----------------------------------------------------------------------------------------

\begin{document}

\section*{Homework 3}

\begin{theorem}
    Let $F$ be a monotone function over a complete lattice $(A, \sqsubseteq)$. Then for every non-decreasing sequence of elements $x_0, x_1, \ldots$, we have that $F(\bigsqcup\{x_0, x_1, \ldots\}) \sqsupseteq \bigsqcup\{F(x_i) ~|~ i \in \mathbb{N}\}$.
\end{theorem}
\begin{proof}
    We write $X = F(\bigsqcup\{x_0, x_1, \ldots\})$ and $Y = \bigsqcup\{F(x_i) ~|~ i \in \mathbb{N}\}$. Let us assume the contrary, i.e., that $X \sqsubset Y$. Therefore, there has to be $i \in \mathbb{N}$ such that $F(x_i) \not\sqsubseteq X$. Otherwise $X$ would be an upper bound of $\{F(x_i) ~|~ i \in \mathbb{N}\}$ which would violate the definition of a least upper bound of $Y$. From the definition of an upper bound, we know that $x_i \sqsubseteq \bigsqcup\{x_0, x_1, \ldots\}$. If we apply monotonicity to the last inequality, we get $F(x_i) \sqsubseteq F(\bigsqcup\{x_0, x_1, \ldots\}) = X$ which is a contradiction.
\end{proof}

\begin{remark}
    It is not true that for any monotone function $F$ over a complete lattice $(A, \sqsubseteq)$ and every non-decreasing sequence of elements $x_0, x_1, \ldots$, we have $F(\bigsqcup\{x_0, x_1, \ldots\}) \sqsubseteq \bigsqcup\{F(x_i) ~|~ i \in \mathbb{N}\}$.
\end{remark}
\begin{proof}
    Let us take the set $A = \mathbb{N} \cup \infty_1 \cup \infty_2$ and $\sqsubseteq$ defined naturally for $\mathbb{N}$ and $\mathbb{N} \sqsubset \infty_1 \sqsubset \infty_2$. This is a complete lattice. We define a monotone function $F$ as: $F(x) = \infty_1$ for every $x \in \mathbb{N}$, $F(\infty_1) = F(\infty_2) = \infty_2$.

    If we define a nondecreasing sequence $x_0, x_1, \ldots = \mathbb{N}$, then $$F\left(\bigsqcup\{x_0, x_1, \ldots\}\right) = F(\infty_1) = \infty_2 \not\sqsubseteq \infty_1 = \bigsqcup \{\infty_1\} = \bigsqcup\{F(x_i) ~|~ i \in \mathbb{N}\}.$$
\end{proof}

\end{document}
